\documentclass[12pt,a4paper]{report}  %紙張設定
\usepackage{xeCJK}%中文字體模組
%\setCJKmainfont{標楷體} %設定中文字體
\setCJKmainfont{MoeStandardKai.ttf}
%\newfontfamily\sectionef{Times New Roman}%設定英文字體
\newfontfamily\sectionef{Nimbus Roman}
\usepackage{enumerate}
\usepackage{titlesec}
\titleformat{\chapter}[display]
{\normalfont\fontsize{20}{22}\selectfont\bfseries\filcenter}
{\chaptertitlename\ \thechapter}{10pt}{\fontsize{18}{22}\selectfont}
\titlespacing*{\chapter}{0pt}{-10pt}{2pt}
\usepackage{amsmath,amssymb}%數學公式、符號
\usepackage{amsfonts} %數學簍空的英文字
\usepackage{graphicx, subfigure}%圖形
\usepackage{fontawesome5} %引用icon
\usepackage{type1cm} %調整字體絕對大小
\usepackage{textpos} %設定文字絕對位置
\usepackage[top=2.5truecm,bottom=2.5truecm,
left=3truecm,right=2.5truecm]{geometry}
\usepackage{titlesec} %目錄標題設定模組
\usepackage{titletoc} %目錄內容設定模組
\usepackage{textcomp} %表格設定模組
\usepackage{multirow} %合併行
%\usepackage{multicol} %合併欄
\usepackage{CJK} %中文模組
\usepackage{CJKnumb} %中文數字模組
\usepackage{wallpaper} %浮水印
\usepackage{listings} %引用程式碼
\usepackage{hyperref} %引用url連結
\usepackage{setspace}
\usepackage{lscape}%設定橫式
\lstset{language=Python, %設定語言
		basicstyle=\fontsize{10pt}{2pt}\selectfont, %設定程式內文字體大小
		frame=lines,	%設定程式框架為線
}
%\usepackage{subcaption}%副圖標
\graphicspath{{./../images/}} %圖片預設讀取路徑
\usepackage{indentfirst} %設定開頭縮排模組
\renewcommand{\figurename}{\Large 圖.} %更改圖片標題名稱
\renewcommand{\tablename}{\Large 表.}
\renewcommand{\lstlistingname}{\Large 程式.} %設定程式標示名稱
\hoffset=-5mm %調整左右邊界
\voffset=-8mm %調整上下邊界
\setlength{\parindent}{3em}%設定首行行距縮排
\usepackage{appendix} %附錄
\usepackage{diagbox}%引用表格
\usepackage{multirow}%表格置中

%--------------------封面-----------------------------%
\begin{document}

\begin{center}
\vspace*{1cm}
  \fontsize{18}{16}\selectfont \textbf{POLITECNICO DI TORINO}\par
\end{center}
\begin{center}
    \fontsize{18}{16}\selectfont \textbf{都靈理工大學}\\
\end{center}
% 在图像下方添加分界线
\noindent\rule{\textwidth}{0.4pt}
\vspace{0.5em}
\begin{center}
  \fontsize{12}{16}\selectfont \textbf{ANALYSIS OF THE ODOO SOFTWARE CAPABILITIES REGARDING 
PRODUCT LIFECYCLE MANAGEMENT, MANUFACTURING EXECUTION 
SYSTEMS AND THEIR INTEGRATION}\par
\end{center}
\begin{center}
   \fontsize{12}{22}\selectfont \textbf{ODOO 軟體功能分析產品生命週期管理、製造執行系統及其集成}
\end{center}
\vspace{1em}
\begin{figure}[h]
\vspace{2cm}
    \centering
    \includegraphics[width=0.33\textwidth]{logo} 
    \label{fig:logo}
\end{figure}
\vspace{2cm}
\noindent
\small\textbf{SUPERVISORS} \hfill \textbf{CANDIDATE}\\
\small\textbf{指導者} \hfill \textbf{申請人}\\
\small\textbf{Giulia Bruno } \hfill \textbf{Lucas Flabiano Perotti}\\
\small\textbf{朱莉婭·布魯諾} \hfill \textbf{盧卡斯·弗拉比亞諾·佩羅蒂}\\
\small\textbf{Franco Lombardi}\hspace{\fill} \\
\small\textbf{佛朗哥·隆巴迪}\hspace{\fill} \\
\noindent\rule{\textwidth}{0.4pt}
\begin{center}
\fontsize{12}{22}\selectfont {Academic Year 2020 – 2021}\\
\fontsize{12}{22}\selectfont {2020 – 2021 學年}
\end{center}
\thispagestyle{empty}
\newpage
%--------------------分頁-----------------------------%
\vspace*{\fill}
\vspace*{\fill}
\begin{raggedright}
    \fontsize{12}{14}\selectfont This work is subject to the Creative Commons Licence\\
    \fontsize{12}{14}\selectfont 本作品受知識共享授權約束\\[2ex]
    \fontsize{12}{12}\selectfont All Rights Reserved\\
    \fontsize{12}{12}\selectfont 版權所有\\

\end{raggedright}
\thispagestyle{empty}
\newpage
%--------------------致謝-----------------------------%
\renewcommand{\baselinestretch}{1.5} %設定行距
\pagenumbering{roman} %設定頁數為羅馬數字
\clearpage  %設定頁數開始編譯
\sectionef
\addcontentsline{toc}{chapter}{ACKNOWLEDGMENTS} %將摘要加入目錄
\begin{center}
\LARGE\textbf{ACKNOWLEDGMENTS}\\
\LARGE\textbf{致謝}\\
\end{center}
\fontsize{14pt}{2.5pt}\sectionef
  {  I would like to thank Dr. Giulia Bruno for her expert advice and invitation to develop this project, as well as Emiliano Traini, for his extraordinary support in this thesis process.}。\\[1pt]

\fontsize{14pt}{5pt}\sectionef
  {  我要感謝朱莉婭·布魯諾博士的專家建議和邀請來開發這個
專案以及埃米利亞諾·特拉尼,感謝他在本論文過程中的非凡支持。}\\[15pt]

\fontsize{14pt}{2.5pt}\sectionef
  {  My most sincere gratitude to my parents, Julio and Michelle, who gave me everything, 
from my life to their extensive and unconditional support and encouragement; also, to my 
brothers and my fiancée Ana, who inspired me through all these years.}\\[1pt]

\fontsize{14pt}{5pt}\sectionef
  {  我最誠摯的感謝我的父母Julio和Michelle,他們給了我一切,從我的生活到他們廣泛、無條件的支持和鼓勵; 還有,對我的
兄弟們和我的未婚妻安娜這些年來一直激勵著我。}\\[15pt]

\fontsize{14pt}{2.5pt}\sectionef
{  My deepest thanks and appreciation to Icaro, Matt, and Maz, for their endless help and 
support throughout not just this project, but for all the other moments in which they pushed 
me to be better. Also, for those who have touched my life, being my greatest gifts, you all 
know who you are, and I am truly grateful for sharing special moments of my lif}\\[1pt]

\fontsize{14pt}{5pt}\sectionef
  {我對 Icaro、Matt 和 Maz 表示最深切的感謝和讚賞,感謝他們的無盡幫助,
不僅支持這個項目,還支持他們推動的所有時刻
我要變得更好。 另外,對於那些感動我生命的人,你們是我最好的禮物
知道你是誰,我真的很感激分享我生命中的特殊時刻}\\[15pt]

\newpage
%--------------------摘要-----------------------------%
\renewcommand{\baselinestretch}{1.5} %設定行距
\pagenumbering{roman} %設定頁數為羅馬數字
\setcounter{page}{2}
\sectionef
\addcontentsline{toc}{chapter}{ABSTRACT} %將摘要加入目錄
\begin{center}
\LARGE\textbf{ABSTRACT}\\
\LARGE\textbf{摘要}\\
\end{center}
\fontsize{14}{18}\sectionef \textbf
 {ANALYSIS  OF  THE ODOO  SOFTWARE  CAPABILITIES  REGARDING  
 PRODUCT  LIFECYCLE  MANAGEMENT,  MANUFACTURING  EXECUTION 
 SYSTEMS  AND  THEIR  INTEGRATION }。\\[2pt]
\fontsize{16}{18}\sectionef \textbf
  {(ODOO 軟體功能分析產品生命週期管理、製造執行系統及其集成)}。\\[15pt]

\fontsize{14pt}{2.5pt}\sectionef 
{ The second half of the 20th century had been marked for the advancements of computer 
technology in all aspects of production.}。\\[1pt]

\fontsize{14pt}{5pt}\sectionef
 {20世紀下半葉以電腦的進步為標誌生產各個環節的技術}\\[15pt]

\fontsize{14pt}{2.5pt}\sectionef 
{ The key feature of that statement is the undeniable truth that alongside the increased 
complexity allowed by computing power comes an ever increasing production of 
overwhelming amounts of information. }。\\[1pt]

\fontsize{14pt}{5pt}\sectionef
 {該聲明的關鍵特徵是不可否認的事實,即隨著增加的計算能力所允許的複雜性,帶來了不斷增加的產量與海量資訊。}\\[15pt]

\fontsize{14pt}{2.5pt}\sectionef 
{From separate perspectives of the industrial landscape, several systems were brewed by 
that sheer necessity for organization, automation and waste reduction focusing on that pool 
of useful data. }。\\[1pt]

\fontsize{14pt}{5pt}\sectionef
 {從工業景觀的不同角度來看,出於組織、自動化和減少浪費的絕對必要性,一些系統誕生了,這些系統專注於有用資料池。}\\[15pt]

\fontsize{14pt}{2.5pt}\sectionef 
{ERP (from a managerial perspective), MES (from a production perspective) and more 
recently PLM (from a strategic development/redevelopment perspective) emerged as 
information solutions tackling this problem from different angles. These solutions, however 
effective, are always plagued by the fundamental incompatibility between the tools that 
implement those systems.}。\\[1pt]

\fontsize{14pt}{5pt}\sectionef
 {ERP(從管理角度)、MES(從生產角度)以及最近的 PLM(從策略開發/再開發角度)作為
資訊解決方案從不同角度解決這個問題。 這些解決方案無論多麼有效,總是受到實現這些系統的工具之間根本不相容的困擾。}\\[15pt]

\fontsize{14pt}{2.5pt}\sectionef 
{This paper objectives revolve around analyzing the integration PLM and MES systems 
from a theoretical perspective and comment on the use of the Odoo software tool to 
implement said integration.}。\\[1pt]

\fontsize{14pt}{5pt}\sectionef
 {本文的目標是從理論角度分析 PLM 和 MES 系統的集成,並對使用 Odoo 軟體工具實現所述集成進行評論。}\\[15pt]

\fontsize{14pt}{2.5pt}\sectionef 
{The Odoo software was described in detail (regarding its use for manufacturing 
envirorment) icluding how it implements PLM and MES. Then, the software was subjected 
to the simulation of a fictional firm devised in the molds of Industry 4.0. This company was
a fictional recently founded small case manufacturing company that uses plastic injection 
molding as their primary mean of production and uses additive manufacturing and fast 
prototyping as part of their business strategy.}。\\[1pt]

\fontsize{14pt}{5pt}\sectionef
 {詳細描述了 Odoo 軟體(關於其在製造環境中的使用),包括它如何實施 PLM 和 MES。 然後,該軟體對一家按照工業 4.0 模式設計的虛構公司進行模擬。 該公司是一家虛構的最近成立的小型箱體製造公司,使用塑膠注塑作為主要生產手段,並使用積層製造和快速原型製作作為其業務策略的一部分。}\\[15pt]

\fontsize{14pt}{2.5pt}\sectionef 
{Keywords: Product Life-Cycle Management, Product Life-Cycle Management, Odoo}。\\[1pt]

\fontsize{14pt}{5pt}\sectionef
 {關鍵字:產品生命週期管理、產品生命週期管理、Odoo}\\[15pt]

\newpage
%--------------------目錄-----------------------------%
%--------------------正文-----------------------------%
\chapter{INTRODUCTION} 
\pagenumbering{arabic} %設定頁號阿拉伯數字
\setcounter{page}{1}  %設定頁數
\begin{center}
\fontsize{18}{16}\selectfont \textbf{介紹}\\
\end{center}
\fontsize{12pt}{2.5pt}\sectionef
\section{Objective目的}
\fontsize{12}{2.5pt}\selectfont {The thesis has the objective of finding out how far PLM+MES system can be implemented 
by using the readily available Odoo software by analyzing the different concepts and 
dynamics that would consist said integration and they apply a fictional scenario to determine 
if and which of those concepts are included within this packaged solution.}\\[1pt]

\fontsize{12}{2.5pt}\selectfont {本論文的目的是透過分析包含所述整合的不同概念和動態,找出使用現成的 Odoo 軟體可以實現 PLM+MES 系統的程度,並應用一個虛構的場景來確定這些概念是否以及哪些概念包含在此打包解決方案中。}\\[15pt]

\fontsize{12}{2.5pt}\selectfont {To contextualize, the Odoo software differs from other solutions in the market 
substantially both in implementation and business model. To summarize, the Odoo software 
was originated as an open-source ERP software as oppose to a PLM or MES software and as 
such its availability and modularity are reasonably expanded. It goes without saying that the 
counter point for this that its usability in the field of PLM or MES is uncertain hence the 
value of this work.}\\[1pt]

\fontsize{12}{2.5pt}\selectfont {從具體情況來看,Odoo 軟體在實施和業務模式方面與市場上的其他解決方案有很大不同。 總而言之,Odoo 軟體最初是一種開源 ERP 軟體,而不是 PLM 或 MES 軟體,因此其可用性和模組化得到了合理擴展。 不言而喻,與之相反的是,它在 PLM 或 MES 領域的可用性是不確定的,因此這項工作的價值。}\\[15pt]

\fontsize{12}{2.5pt}\selectfont {Specifically, from the perspective of small manufacturing business and startups, the idea 
of an all-around ERP that implements a PLM-MES system is extremely valuable. Although
ERP systems are somewhat available, they rarely venture deep enough into manufacturing to 
expand into PLM or MES solutions. In addition, the other direction is also relevant since 
PLM solutions tend to not have the expandability of an ERP which usually means that any 
integration requires specialized ad-hoc work.}\\[1pt]

\fontsize{12}{2.5pt}\selectfont {具體來說,從小型製造企業和新創企業的角度來看,實施PLM-MES系統的全能ERP的想法非常有價值。 儘管 ERP 系統在一定程度上可用,但它們很少深入製造業以擴展到 PLM 或 MES 解決方案。 此外,另一個方向也相關,因為 PLM 解決方案往往不具備 ERP 的可擴展性,這通常意味著任何整合都需要專門的臨時工作。}\\[15pt]

\fontsize{12}{2.5pt}\selectfont {Although modifying the software do not fall within the scope of this work, the fact that the software has an open-source community version means that adapting the software even to the most specific cases may prove to be easier and economical barriers for adopting lower,further emphasizing the possible utility of this software in the context of small business.}\\[1pt]

\fontsize{12}{2.5pt}\selectfont {儘管修改軟體不屬於這項工作的範圍,但該軟體具有開源社群版本這一事實意味著,甚至使該軟體適應最具體的情況也可能被證明是更容易和更經濟的採用更低、更進一步的障礙。強調該軟體在小型企業中的可能實用性。}\\[15pt]

\fontsize{12}{2.5pt}\selectfont {具體來說,從小型製造企業和新創企業的角度來看,實施PLM-MES系統的全能ERP的想法非常有價值。 儘管 ERP 系統在一定程度上可用,但它們很少深入製造業以擴展到 PLM 或 MES 解決方案。 此外,另一個方向也相關,因為 PLM 解決方案往往不具備 ERP 的可擴展性,這通常意味著任何整合都需要專門的臨時工作。}\\[15pt]
\newpage
\fontsize{12}{2.5pt}\selectfont {Ultimately, the thesis will give theoretical and practical advices on how to further exploit this system. It will also lay the ground for future works on the Odoo software and checks on how the solution is performing by identifying specific key aspects of PLM-MES integration and implementation}\\[1pt]

\fontsize{12}{2.5pt}\selectfont {最終,本文將為如何進一步利用該系統提供理論和實務建議。 它還將為 Odoo 軟體的未來工作奠定基礎,並透過確定 PLM-MES 整合和實施的具體關鍵方面來檢查解決方案的性能}\\[15pt]
\section{Structure結構 }
\fontsize{12}{2.5pt}\selectfont {This work could be a reference for an actual implementation of the described solution in 
small manufacturing enterprises and it can be treated as introductory material to PLM-MES 
and their implementation, as well as first principles and review of the current state of the 
Odoo software regarding it. To such end, this thesis presents the following structure:}\\[1pt]

\fontsize{12}{2.5pt}\selectfont {這項工作可以為小型製造企業中所描述的解決方案的實際實施提供參考,並且可以將其視為 PLM-MES 及其實施的介紹材料,以及 Odoo 軟體的首要原則和當前狀態的回顧它。 為此,本文提出以下結構:}\\[15pt]
\begin{enumerate}[{\hspace{0.5em}\textbullet}]
\fontsize{12}{2.5pt}\selectfont
            \item Chapter 1 - Introduction to this work and its objectives. Furthermore, it provide a 
succinct explanation of why this software solution requires this sort of analysis in the 
first place and how it was be structured.\\
第 1 章 - 介紹這項工作及其目標。 此外,它還簡要解釋了為什麼該軟體解決方案首先需要進行此類分析以及它是如何建構的。
\item Chapter 2 – This chapter introduce the basic theoretical background to PLM, MES, 
ERP and Industry 4.0. These are presented in order to create the grounds to a 
meaningful contribution in this kind of analysis as well as providing meaningful 
context for its implementation in case the reader is a small business representative.\\
第 2 章 – 本章介紹 PLM、MES、ERP 和工業 4.0 的基本理論背景。 提出這些內容是為了為此類分析做出有意義的貢獻奠定基礎,並為讀者是小型企業代表的情況下的實施提供有意義的背景。
\item Chapter 3 – This chapter is all about the integration between PLM and MES systems 
as discussed by previous works and as was be analyzed in this work. This is useful to 
stablish the concepts and dynamics that are the subject when analyzing the Odoo
software.\\
第 3 章 – 本章主要介紹 PLM 和 MES 系統之間的集成,如先前的工作所討論的和本工作中所分析的。 這對於在分析 Odoo 軟體時確定主題的概念和動態很有用。
\item Chapter 4 – Introduction to the fictional company and products chosen in the molds 
of Industry 4.0 to be used in the further analysis and evaluation of the Odoo software.\\
第 4 章 – 介紹在工業 4.0 模型中選擇的虛構公司和產品,用於進一步分析和評估 Odoo 軟體。
\item  Chapter 5 – The introduction to the Odoo software as well as a more in-depth
explanation of its use and functionalities. The description of the experimentation of 
the Odoo software taking in consideration all the previous chapters.\\
第 5 章 – Odoo 軟體簡介以及對其使用和功能的更深入解釋。 考慮到前面所有章節的 Odoo 軟體實驗描述
\item Chapter 7 - Conclusions The last chapter describes the takeaways of the work: how a 
medium enterprise can improve its processes through an informed use of a 
PLM+MES system implemented using the Odoo software.\\
第 7 章 - 結論 最後一章介紹了工作的要點:中型企業如何透過明智地使用使用 Odoo 軟體實施的 PLM+MES 系統來改善其流程。

\end{enumerate}
\renewcommand{\baselinestretch}{0.5} %設定行距

\newpage
\input{02.tex}
\newpage
\chapter{THE STATE OF THE ART AND THE INTEGRATION OF PLM AND MES} 
\pagenumbering{arabic} %設定頁號阿拉伯數字
\setcounter{page}{17}  %設定頁數
\begin{center}
\fontsize{18}{16}\selectfont \textbf{最先進的技術以及 PLM 和 MES 的集成}\\
\end{center}
\vspace{2em}

\fontsize{14pt}{2.5pt}\sectionef 
{Unfortunately, there are not many published studies in the matter of integration between PLM and MES systems. But there seems to be a consensus in the most probable effects of said integration. Those being synchronization and tighter tolerances.}\\[10pt]

\fontsize{14pt}{5pt}\sectionef
 {遺憾的是,關於 PLM 和 MES 系統整合問題的已發表研究並不多。但對於上述整合最可能產生的影響似乎已達成共識。這些是同步和更嚴格的公差。}\\[15pt]

\fontsize{14pt}{2.5pt}\sectionef 
{As explained by D’Antonio et al. (2015), which focus on a case study involving the manufacturing of precision components for aeronautical applications, the first advantage expected by the deployment of the monitoring and control system is product quality improvement: sensors allow to detect, measure and monitor variables, events and situations that affect process performance or product quality.}\\[10pt]

\fontsize{14pt}{5pt}\sectionef
 {正如德安東尼奧等人所解釋的。 (2015),重點在於涉及航空應用精密零件製造的案例研究,部署監控系統預期的第一個優勢是產品品質改進:感測器允許檢測、測量和監控變數、事件和影響製程性能或產品品質的情況。
}\\[15pt]

\fontsize{14pt}{2.5pt}\sectionef 
{One of the central problems regarding integrating PLM with any other system revolves around the ownership of information. A possible solution relies on database integration as well as the use of middleware between systems. As is written in Saaksvuori and Immonen, (2008). A reasonable objective is that information should always be updated in one place. Other systems can read information directly from the PLM databases, and if necessary, the required information can be replicated on the databases of other system, as depicted in Figure7. Although it points this out mainly from the perspective of PLM-ERP integration, it is still very valuable from the perspective of PLM-MES integration because it is an example of how the better operation can be expected by working around systems in which files of different nature are loaded into a centralized PLM-ERP system.}\\[10pt]

\fontsize{14pt}{5pt}\sectionef
 {將 PLM 與任何其他系統整合的核心問題之一涉及資訊的所有權。一種可能的解決方案依賴於資料庫整合以及系統之間中間件的使用。正如 Saaksvuori 和 Immonen (2008) 所寫。一個合理的目標是資訊應始終在一個地方更新。其他系統可以直接從PLM資料庫中讀取訊息,如果需要,可以將所需資訊複製到其他系統的資料庫上,如圖7所示。雖然主要是從PLM-ERP整合的角度指出了這一點,但從PLM-MES 整合的角度來看,仍然非常有價值,因為它是一個範例,說明如何透過將不同性質的檔案載入到集中式PLM-ERP 系統中的系統來實現更好的操作。
}\\[15pt]
\newpage
\begin{figure}[hbt!]
\begin{center}
\includegraphics[width=10cm]{7}
\caption{\Large Diagram of PLM integration PLM整合示意圖}\label{fig.7}
\end{center}
\end{figure}

\fontsize{14pt}{2.5pt}\sectionef 
{The middleware would therefore be a software framework to organize and connect all the information given to the system database in a user-friendly way. This sort of application is also referred to as integration application and, as specified by Stark (2015), these applications enable exchange of product information between PLM applications (for example, between a CAD application and a CAE application). They also enable exchange of product information between PLM applications and other enterprise applications such as ERP and CRM.}\\[10pt]

\fontsize{14pt}{5pt}\sectionef
 {因此,中間件將成為一個軟體框架,以使用者友好的方式組織和連接提供給系統資料庫的所有資訊。此類應用程式也稱為整合應用程序,並且根據 Stark (2015) 的規定,這些應用程式支援在 PLM 應用程式之間(例如 CAD 應用程式和 CAE 應用程式之間)交換產品資訊。它們還支援 PLM 應用程式與其他企業應用程式(例如 ERP 和 CRM)之間的產品資訊交換。}\\[15pt]

\fontsize{14pt}{2.5pt}\sectionef 
{In a very relevant fashion, this middleware line of thinking is expanded upon by (Ben Khedher et al., 2011). In their work regarding different systems architectures for the implementation of an integrated MES+PLM they describe the use of a mediation system in web service architecture. As depicted in Figure 8, the proposed architecture uses data exchange based on internet technologies to help companies, especially expanded companies, to take advantage of opportunities generated by the Web Services. The concept of "web service" means an application (program or software system) which is designed to support interoperable machine-to-machine interactions over a network, according to the definition of W3C (Ben Khedher et al., 2011).}\\[10pt]

\fontsize{14pt}{5pt}\sectionef
 {(Ben Khedher et al., 2011) 以非常相關的方式擴展了這種中間件思路。在他們關於實施整合 MES+PLM 的不同系統架構的工作中,他們描述了中介系統在 Web 服務架構中的使用。如圖 8 所示,所提出的架構使用基於互聯網技術的資料交換來幫助公司,特別是擴張型公司,利用 Web 服務產生的機會。根據 W3C 的定義,「Web 服務」的概念是指旨在支援網路上可互通的機器對機器互動的應用程式(程式或軟體系統)(Ben Khedher 等,2011)。}\\[15pt]
\newpage

\begin{figure}[hbt!]
\begin{center}
\includegraphics[width=15cm]{8}
\caption{\Large Diagram of Web service architecture Web服務架構圖}\label{fig.8}
\end{center}
\end{figure}

\fontsize{14pt}{2.5pt}\sectionef 
{The reason this expansion is so relevant from the perspective of this work is that the Odoo software works in a similar fashion through a similar web service architecture. In theory the Odoo software could act as the middleware working through the local network or hosted in the cloud and enacting the layer of integration that was previously mentioned.}\\[10pt]

\fontsize{14pt}{5pt}\sectionef
 {從這項工作的角度來看,這種擴展如此相關的原因是 Odoo 軟體透過類似的 Web 服務架構以類似的方式運作。理論上,Odoo 軟體可以充當透過本地網路工作或託管在雲端中的中間件,並執行前面提到的整合層。}\\[15pt]

\section{How would this integration look like in practical terms 這種整合在實際中會是什麼樣子}

\fontsize{14pt}{2.5pt}\sectionef 
{As mentioned in CHAPTER 2 the main idea of PLM is to manage change in all processes related to the product, and it does so mainly through the use of virtualization. The word virtualization here denotes representation of item of the real world to the digital space and, as one can imagine, there are several levels of abstraction through which a real object or process can be represented. As consequence there is no exact consensus regarding PLM of how deep and/or detailed the virtual representation must be to serve its purpose.}\\[10pt]

\fontsize{14pt}{5pt}\sectionef
 {如第 2 章中所提到的,PLM 的主要想法是管理與產品相關的所有流程中的變更,它主要透過使用虛擬化來實現。這裡的「虛擬化」一詞表示現實世界的項目在數位空間中的表示,並且正如人們可以想像的那樣,可以透過多個抽象層級來表示真實的物件或過程。因此,對於 PLM 虛擬表示必須有多深和/或多詳細才能達到其目的,還沒有達成確切的共識。}\\[15pt]

\fontsize{14pt}{2.5pt}\sectionef 
{In an ideal world that would be the lowest form of abstraction which, essentially, would come down to a digital twin as explained in the CHAPTER 2. This is a ‘1 to 1’ digital representation of every aspect of the production cycle where every part involved would have a digital representation that not only carry the physical characteristics of the item but also all its information produced over time. To this end, as explained in CHAPTER 2, MES takes a fundamental role in obtaining the real time information required for the DT even be possible.}\\[10pt]

\fontsize{14pt}{5pt}\sectionef
 {在理想的世界中,這將是最低的抽象形式,本質上將歸結為數位孿生,如第 2 章所述。這是生產週期各個方面的「1 對 1」數字表示,其中每個部分所涉及的內容將具有數位表示,不僅包含該項目的物理特徵,還包含隨著時間的推移產生的所有資訊。為此,如第 2 章所述,MES 在獲取 DT 所需的即時資訊方面發揮基礎作用,甚至是可能的。}\\[15pt]

\fontsize{14pt}{2.5pt}\sectionef 
{For instance, a CNC machine would have a digital 3D model for simulation as well as a fully integrated list of all the pieces it produces, data regarding its current level of production, the current wear of its mechanical pieces, all other machines it relates to, history of all the alterations and improvements by which it was affected and many other aspects, all well packaged in an intuitive graphical user interface (GUI) that allows for maximum interaction.}\\[10pt]

\fontsize{14pt}{5pt}\sectionef
 {例如,數控機床將具有用於模擬的數位 3D 模型以及其生產的所有零件的完全整合清單、有關其當前生產水平的數據、其機械零件的當前磨損情況以及與其相關的所有其他機器、受影響的所有變更和改進的歷史記錄以及許多其他方面,全部都很好地封裝在直覺的圖形使用者介面(GUI) 中,可實現最大程度的互動。}\\[15pt]

\fontsize{14pt}{2.5pt}\sectionef 
{Outside of fiction, we are yet to achieve such level of virtualization. It takes too much time and money to obtain and organize information to such a level of minutia, specially, the aspects that need to be inserted by hand, not to mention the subjectiveness of how this information can be integrated and interacted with. Regardless of that it is useful to identify, within the ideal, the aspects of most importance for this implementation.}\\[10pt]

\fontsize{14pt}{5pt}\sectionef
 {除了小說之外,我們還沒有達到這樣的虛擬化程度。獲取和組織如此詳細的資訊需要花費太多的時間和金錢,特別是需要手動插入的方面,更不用說如何整合和互動這些資訊的主觀性了。不管怎樣,在理想情況下確定對於此實施最重要的方面是有用的。}\\[15pt]

\fontsize{12}{2.5pt}\selectfont {Those are:}\\[1pt]
\fontsize{12}{2.5pt}\selectfont {那些是:}\\[15pt]

\begin{enumerate}[{\hspace{0.5em}\textbullet}]
\fontsize{12}{2.5pt}\selectfont
\item The means of virtualization – What sort of information is used to build the virtual items. This includes the metadata and files that are directly attached to the item. In an ideal fashion this would contain all possible information available about the item.\\
虛擬化的手段-使用什麼樣的資訊來建構虛擬物品。這包括直接附加到項目的元資料和文件。在理想的情況下,這將包含有關該項目的所有可能的可用資訊。
\item The means of data input - How this information is being loaded and organized. Ideally this information would be loaded into the system as automatically as possible, be it by means of MES during quality control or through the use of automated input tools like bar code scanners.\\
資料輸入的方式 - 如何載入和組織這些資訊。理想情況下,這些資訊將盡可能自動載入到系統中,無論是在品質控制期間透過 MES 還是透過使用條碼掃描器等自動輸入工具。
\item The means of access – How this information is presented to the users. Although more subjective than the previous aspects this is incredibly important to the way the system is interacted with. How intuitive it is the information availability plays right into the core strengths of PLM. Afterall, everything would be for nothing (even if all else would be perfect) if the only way to interact with the system were a command line interface that would make difficult for the end users to access the information.\\
存取方式-如何將資訊呈現給使用者。儘管比前面的方面更主觀,但這對於系統互動的方式非常重要。資訊可用性的直覺程度正是 PLM 的核心優勢。畢竟,如果與系統互動的唯一方式是命令列介面,而這將使最終用戶存取資訊變得困難,那麼一切都將毫無意義(即使其他一切都很完美)。
\item The means of integration - How items and their contained information can interact and benefit from one another, i.e., the integration with other systems and key softwares. E.g., if an item has access to a cad file, there should be no need to fill in the metadata fields by hand. Hoe items can automatically affect other items also plays into this aspect.\\
整合方式 - 專案及其所包含的資訊如何互動並相互受益,即與其他系統和關鍵軟體的整合。例如,如果某個項目可以存取 cad 文件,則無需手動填寫元資料欄位。鋤頭項目可以自動影響其他項目也能發揮作用。

\end{enumerate}
\renewcommand{\baselinestretch}{0.5} %設定行距
\newpage
\chapter{INTRODUCTION TO THE COMPANY AND PRODUCT} 
\pagenumbering{arabic} %設定頁號阿拉伯數字
\setcounter{page}{22}  %設定頁數
\begin{center}
\fontsize{18}{16}\selectfont \textbf{公司及產品介紹}\\
\end{center}
\vspace{1em}

\fontsize{14pt}{2.5pt}\sectionef 
{As one can imagine, one of the unique aspects of this work is its focus in one specific software solution that tend to be quite flexible in terms of ease of implementation to different sorts of business. This is contrary to most use cases regarding PLM implementation where the business case is the constant and the system is built around it. Nonetheless, in order to evaluate Odoo as a PLM+MES tool, it is important to consider an example. The advantage here is that a fictional company can be picked for this end maximizing the perceived effect of the software during a simulation.}\\[5pt]

\fontsize{14pt}{5pt}\sectionef
 {可以想像,這項工作的獨特之處之一是它專注於一個特定的軟體解決方案,該解決方案在易於實施不同類型的業務方面往往非常靈活。這與大多數有關 PLM 實施的用例相反,在這些用例中,業務案例是不變的,系統是圍繞它構建的。儘管如此,為了評估 Odoo 作為 PLM+MES 工具的能力,考慮一個範例很重要。這裡的優點是可以選擇一家虛構的公司來實現這一目的,從而在模擬過程中最大化軟體的感知效果。}\\[15pt]

\fontsize{14pt}{2.5pt}\sectionef 
{It is considering all those previously mentioned systems that, for the sake of exemplification, the theoretical company was organized in the molds of Industry 4.0. This company is a recently founded small case manufacturing company that uses plastic injection molding as their primary mean of production and uses additive manufacturing and fast prototyping as part of their business strategy. As explained in chapter 2 those are great examples of the path that industry is taking regarding innovation where mass production is becoming slowly less important than product variety and time to market.}\\[5pt]

\fontsize{14pt}{5pt}\sectionef
 {考慮到前面提到的所有系統,為了舉例說明,理論公司是按照工業 4.0 的模式組織的。該公司是一家最近成立的小型箱體製造公司,使用塑膠射出成型作為主要生產方式,並使用積層製造和快速原型製作作為其業務策略的一部分。如第 2 章所解釋的,這些都是產業在創新方面所採取的路徑的很好的例子,在這種創新中,大規模生產逐漸變得不如產品品種和上市時間重要。}\\[15pt]

\fontsize{14pt}{2.5pt}\sectionef 
{In order to maximize the tracking of change, most of its business are based on lower production batches on mainly automated machinery. This company focus in the production of injected plastic products and rely heavily in flexible machinery for setting production and prototyping. Having that in mind, it should be simple enough to simulate continuous improvement of both product and process to the extent of the evaluated software. Since this sort of everchanging production is extremely dependent on information management of all kinds, it must prove to be a perfect base for applied PLM+MES.}\\[5pt]

\fontsize{14pt}{5pt}\sectionef
 {為了最大限度地追蹤變化,其大部分業務都基於主要自動化機械上的較低生產批次。該公司專注於注射塑膠產品的生產,並嚴重依賴靈活的機械來進行生產和原型製作。考慮到這一點,它應該足夠簡單,可以在評估的軟體範圍內模擬產品和流程的持續改進。由於這種不斷變化的生產極其依賴各種資訊管理,因此它必須成為應用 PLM+MES 的完美基礎。}\\[15pt]

\fontsize{14pt}{2.5pt}\sectionef 
{In this example the company has already implemented, since its recent foundation, the Odoo software and has taken all the necessary training and steps to its proper use. This allow the removal of the boundaries and limitations that are so common regarding implementation of the PLM+MES system to an already existing business, i.e., dependences on legacy systems administrative resistance to change or integration to old procedures. These are obviously important, but it is not within the scope of this work.}\\[10pt]

\fontsize{14pt}{5pt}\sectionef
 {在這個例子中,該公司自最近成立以來已經實施了 Odoo 軟體,並採取了所有必要的培訓和步驟來正確使用該軟體。這樣可以消除在現有業務中實施 PLM+MES 系統時常見的邊界和限制,即依賴遺留系統管理對更改或整合舊程式的抵制。這些顯然很重要,但不屬於本工作的範圍。}\\[15pt]

\fontsize{14pt}{2.5pt}\sectionef 
{The company aims to produce a completely new product by the end of the year. After doing so, the company improved the process of production for said product. Once there is the need for product improvement, said improvement was performed as well.}\\[10pt]

\fontsize{14pt}{5pt}\sectionef
 {該公司的目標是在今年底前生產出全新的產品。在此之後,該公司改進了該產品的生產流程。一旦產品需要改進,也進行改進。}\\[15pt]

\fontsize{14pt}{2.5pt}\sectionef 
{The following diagram (Figure 9) will be taken into consideration as the path of product development and improvement:}\\[10pt]

\fontsize{14pt}{5pt}\sectionef
 {將考慮下圖(圖9)作為產品開發和改進的路徑:}\\[15pt]

\begin{figure}[hbt!]
\begin{center}
\includegraphics[width=15cm]{9}
\caption{\Large Development diagram 開發圖}\label{fig.9}
\end{center}
\end{figure}

\fontsize{14pt}{2.5pt}\sectionef 
{This path aims to transmit to the reader an iterative approach towards development and improvement. The idea is followed by a product design for which a cycle of prototyping and redesign takes effect until satisfactory result is achieved. Then a similar cycle takes place regarding the production process. At the end of this stage initial development is done and the actual production can begin.}\\[10pt]

\fontsize{14pt}{5pt}\sectionef
 {這條路徑旨在向讀者傳達一種開發和改進的迭代方法。這個想法之後是產品設計,原型製作和重新設計的循環生效,直到獲得滿意的結果。然後在生產過程中會發生類似的循環。在此階段結束時,初步開發完成,實際生產即可開始。}\\[15pt]

\fontsize{14pt}{2.5pt}\sectionef 
{It is at this point that ways of stablishing the continuous improvement is important. In the case of this company, we are only considering two main types of upgrade paths, those being, product upgrade and process upgrade respectively.}\\[10pt]

\fontsize{14pt}{5pt}\sectionef
 {正是在這一點上,建立持續改進的方法很重要。就這家公司而言,我們只考慮兩個主要的升級路徑,分別是產品升級和流程升級。}\\[15pt]

\section{The products and processes 產品和工藝}

\fontsize{14pt}{2.5pt}\sectionef 
{Change and effect are the focus of the PLM+MES implementation as such the subject of said change would ideally be something that could afford a reasonable amount of freedom of design. Although the effects of a well implemented PLM+MES should be substantial even in rigid manufacturing environments, where the change is extremely limited, the system will produce much more perceivable change in an enterprise that thrives in innovation because there will be more opportunities to improve the system and gain feedback.}\\[10pt]

\fontsize{14pt}{5pt}\sectionef
 {變更和效果是 PLM+MES 實施的重點,因此理想情況下,上述變更的主題應該能夠提供合理的設計自由。雖然實施良好的PLM+MES 的效果即使在變化極其有限的嚴格製造環境中也應該是顯著的,但該系統將在創新蓬勃發展的企業中產生更明顯的變化,因為將有更多的機會來改進系統並獲得回饋。}\\[15pt]

\fontsize{14pt}{2.5pt}\sectionef 
{From the perspective of improvement, if you compare a product that is a result from sheet metal stamping (Figure 10) to an equivalent product that is the result of a CNC milling procedure (Figure 11) it is easy to perceive that the CNC milled product is more welcoming to upgrades. While the stamping is low cost (by comparison) it depends on heavy high precision metal dyes that are extremely expensive to produce. This means that the cost of enacting change to it is much higher and thus the effect of a system that thrives on tracking change becomes limited.}\\[10pt]

\fontsize{14pt}{5pt}\sectionef
 {從改進的角度來看,如果將鈑金沖壓的產品(圖 10)與 CNC 銑削加工的同等產品(圖 11)進行比較,很容易看出 CNC 銑削的產品更歡迎升級。雖然沖壓成本較低(相比之下),但它依賴於生產成本極其昂貴的重型高精度金屬染料。這意味著對其進行更改的成本要高得多,因此依靠追蹤更改而蓬勃發展的系統的效果變得有限。}\\[15pt]
\newpage

\begin{figure}[hbt!]
\begin{center}
\includegraphics[width=15cm]{10}
\caption{\Large Example of stamped AK74 pattern rifle receiver (Brownnells.com) 印有AK74圖案的步槍機匣範例 (Brownnells.com)}\label{fig.10}
\end{center}
\end{figure}

\begin{figure}[hbt!]
\begin{center}
\includegraphics[width=15cm]{11}
\caption{\Large Example of milled AK74 pattern rifle receiver (sharpsbros.com) 銑削 AK74 型步槍機匣範例 (sharpsbros.com)}\label{fig.11}
\end{center}
\end{figure}

\fontsize{14pt}{2.5pt}\sectionef 
{In the case of this fictional company, it has been determined that the best way to exemplify the PLM+MES effects would be to have products designed around plastic injection molding. It might seem unintuitive at first to consider this manufacturing procedure, like the stamping procedure previously described, since it too depends on high precision molds during production. However, the main differences between the two is regarding ease of prototyping and the cost of upgrading.}\\[10pt]

\fontsize{14pt}{5pt}\sectionef
 {就這家虛構的公司而言,我們確定體現 PLM+MES 效果的最佳方式是圍繞塑膠射出成型設計產品。乍一看,考慮這種製造過程(如前面描述的沖壓過程)似乎不直觀,因為它在生產過程中也依賴高精度模具。然而,兩者之間的主要區別在於原型設計的難易度和升級成本。}\\[15pt]

\fontsize{14pt}{2.5pt}\sectionef 
{Injection molding is a broad and complex field of engineering that involves a huge variety of materials and methods, little of which is of the concern of this work. It is however relevant to point out that for the most part, the pressures involved in the injection molding are one order of magnitude lower than the when we are dealing with steel; softer materials can be used on their molds like CNC milled aluminum. At the same time, new advancements in the field of additive manufacturing have made possible to prototype plastic parts with much closer physical characteristics to the end result of a injected piece. Sometimes even prototype molds (Figure 12) can be used for a lower volume test runs during process upgrades.}\\[10pt]

\fontsize{14pt}{5pt}\sectionef
 {射出成型是一個廣泛而複雜的工程領域,涉及各種各樣的材料和方法,但本工作只涉及其中很少一部分。然而,需要指出的是,在大多數情況下,射出成型中涉及的壓力比我們處理鋼時的壓力低一個數量級;模具上可以使用較軟的材料,例如 CNC 銑削鋁。同時,積層製造領域的新進步使得塑膠零件原型成為可能,其物理特性與注射件的最終結果更加接近。有時,甚至原型模具(圖 12)也可用於製程升級期間的小批量試運行。}\\[15pt]

\begin{figure}[hbt!]
\begin{center}
\includegraphics[width=15cm]{12}
\caption{\Large Example of injection mold made using a 3D printer (thefabricator.com) 使用 3D 列印機製作射出成型模具的範例 (thefabricator.com)}\label{fig.12}
\end{center}
\end{figure}
\newpage

\fontsize{14pt}{2.5pt}\sectionef 
{Additive manufacturing has become an incredible tool for ultra-flexible production. This mindset of continuous improvement, especially when regarding prototyping and iterative design, is a hallmark of the lean mentality that is so relevant in the modern industry.}\\[10pt]

\fontsize{14pt}{5pt}\sectionef
 {增材製造已成為超靈活生產的不可思議的工具。這種持續改進的心態,特別是在原型設計和迭代設計方面,是與現代工業密切相關的精益心態的標誌。
}\\[15pt]

\fontsize{14pt}{2.5pt}\sectionef 
{As mentioned in the previous section, in this case study it is considered the creation of a new product and its production process by the fictional company. This product consists in a plastic small form factor computer case, composed of 3 different parts (Figure 13) that are expected to be designed and prototyped considering combination of additive manufacturing and CNC milling towards a plastic injection molding production.}\\[10pt]

\fontsize{14pt}{5pt}\sectionef
 {如上一節所提到的,在本案例研究中,它被視為虛構公司的新產品的創造及其生產過程。該產品由一個塑膠小型電腦機殼組成,由 3 個不同的部件組成(圖 13),預計在設計和原型製作時考慮將增材製造和 CNC 銑削相結合進行塑膠射出成型生產。
}\\[25pt]

\begin{figure}[hbt!]
\begin{center}
\includegraphics[width=15cm]{13}
\caption{\Large 3D exploded view of the theoretical product 理論產品的 3D 分解圖}\label{fig.13}
\end{center}
\end{figure}
\newpage

\subsection{Part A A部分}

\fontsize{14pt}{2.5pt}\sectionef 
{PART-A (Figure 14) is the core structure of the computer case. It is expected to comport all the pieces necessary for the proper function of the small form factor computer in question. To this end a raw material A was selected to be Acrylonitrile Butadiene Styrene (ABS) this is an opaque thermoplastic polymer and an engineering grade plastic. It is commonly used to produce electronic parts such as phone adaptors, keyboard keys and wall socket plastic guards.}\\[15pt]

\fontsize{14pt}{5pt}\sectionef
 {PART-A(圖14)是電腦機殼的核心結構。它預計將配備所討論的小型計算機正常運行所需的所有部件。為此,原料 A 被選為丙烯腈丁二烯苯乙烯 (ABS),這是一種不透明的熱塑性聚合物和工程級塑膠。它通常用於生產電子零件,例如電話適配器、鍵盤按鍵和牆壁插座塑膠防護罩。}\\[15pt]

\begin{figure}[hbt!]
\begin{center}
\includegraphics[width=15cm]{14}
\caption{\Large Isometric view of Part A A 部分等距視圖}\label{fig.14}
\end{center}
\end{figure}
\vspace{1cm}

\fontsize{14pt}{2.5pt}\sectionef 
{The main reasons for choosing this material specifically are its toughness, its good dimensional stability (resistance to change dimensions after cooling), its high impact resistance and surface hardness. Finally, it is also commonly available in the form of 3D printing filament for extrusion 3D printers which should prove to be quite useful during prototyping.}\\[15pt]

\fontsize{14pt}{5pt}\sectionef
 {特別選擇這種材料的主要原因是其韌性、良好的尺寸穩定性(冷卻後不易改變尺寸)、高抗衝擊性和表面硬度。最後,它通常也以用於擠出 3D 列印機的 3D 列印絲的形式提供,這在原型製作過程中應該非常有用。}\\[15pt]
\newpage

\subsection{Part B and C     B 部分和 C 部分}

\fontsize{14pt}{2.5pt}\sectionef 
{Parts B and C are lids that should snap into place, closing the system. These are very simple pieces and require a certain level of elasticity so it can deform to assure a screwless assembly. These two identical parts are going to be made with Thermoplastic Polyurethane (TPU), because of its elastic nature and great tensile and tear strength. This sort of polymer is often used to produce parts that demand a rubber-like elasticity. TPU performs well at high temperatures and is commonly used in power tools, cable insulations and sporting goods. Finally, TPU is also available in the form of filament for 3D printers which, for the simulation, will be used for prototyping.}\\[15pt]

\fontsize{14pt}{5pt}\sectionef
 {B 和 C 部分是蓋子,應卡入到位,關閉系統。這些都是非常簡單的零件,需要一定程度的彈性,以便可以變形以確保無螺絲組裝。這兩個相同的部件將由熱塑性聚氨酯 (TPU) 製成,因為它具有彈性、拉伸強度和撕裂強度。這種聚合物通常用於生產需要類似橡膠彈性的零件。 TPU 在高溫下表現良好,常用於電動工具、電纜絕緣材料和體育用品。最後,TPU 還可以以細絲的形式用於 3D 列印機,用於模擬,用於原型製作。}\\[15pt]

\begin{figure}[hbt!]
\begin{center}
\includegraphics[width=15cm]{15}
\caption{\Large Part B and C     B 部分和 C 部分}\label{fig.15}
\end{center}
\end{figure}

\subsection{Molds 模具}

\fontsize{14pt}{2.5pt}\sectionef 
{Ideally all molds should be made of steel, for longevity of the mold and product quality. That being said, the injected plastics that are being selected for all parts are not so pressure dependent and their forms are not so complex, so it is assumed that aluminum molds made with a precision CNC machining should suffice to produce said parts.}\\[10pt]

\fontsize{14pt}{5pt}\sectionef
 {理想情況下,所有模具應由鋼製成,以確保模具的使用壽命和產品品質。話雖如此,為所有零件選擇的注射塑膠並不那麼依賴壓力,而且它們的形狀也不是那麼複雜,因此假設用精密 CNC 加工製造的鋁模具應該足以生產所述零件。}\\[15pt]

\fontsize{14pt}{2.5pt}\sectionef 
{It is also assumed that all molds are simple enough to be prototyped using 3D printing. Although this is not always true, it was determined representative enough for this simulation. The type of material used in those prototypes is high temperature resign cured using an SLA 3DPrinter. Additionally, the mold will be considered the main physical aspect to be developed when regarding the production process because it something that directly affects the production as well as something that can be produced in house and tracked as a product would.}\\[10pt]

\fontsize{14pt}{5pt}\sectionef
 {也假設所有模具都足夠簡單,可以使用 3D 列印來製作原型。儘管這並不總是正確的,但它對於本次模擬來說具有足夠的代表性。這些原型中使用的材料類型是使用 SLA 3D 列印機進行高溫重新固化的。此外,在生產過程中,模具將被視為要開發的主要物理方面,因為它直接影響生產,並且可以像產品一樣在內部生產和追蹤。}\\[15pt]

\section{What is analized during the simulation 模擬過程中分析了什麼}

\fontsize{14pt}{2.5pt}\sectionef 
{Taking into consideration the diagram, shown in Figure 9, as well as the main aspects of a successful integration of PLM and MES as described in the section 3.1, this experiment aims to produce commentary regarding the following relevant questions in Table 1.}\\[10pt]

\fontsize{14pt}{5pt}\sectionef
 {考慮到圖 9 所示的圖表,以及第 3.1 節中描述的 PLM 和 MES 成功整合的主要方面,本實驗旨在對錶 1 中的以下相關問題進行評論。}\\[15pt]

\begin{center}
 \fontsize{16}{16}\selectfont \textbf{Table 1 Summary of questions to be answered}\\
\fontsize{16}{16}\selectfont \textbf{表1 需回答的問題總表}\\
\end{center}

\begin{figure}[hbt!]
\begin{center}
\includegraphics[width=15cm]{15-2}
\end{center}
\end{figure}

\newpage



\begin{table}[htbp]
    \centering
    \label{tab:example}
    \begin{tabular}{|c|c|}
        \hline
        類別 & 問題 \\
        \hline
        軟體如何處理物品? & 是否代表了產品生命週期的所有面向? \\
         & 這些項目的表現如何?\\
         \hline
        創造一個全新的產品有多容易? & 產品的描述方式? \\
         & 產品如何整合和引用相關文件?\\
         & 改變一個會影響另一個嗎? \\
         \hline
       創造一個是多麼容易全新的生產流程? & 其過程是如何描述的? \\
         & 該流程如何整合和參考其生產的產品?\\
         & 改變一個會影響另一個嗎? \\
         \hline
        改進現有產品有多容易? & 更新元數據有多容易? \\
         & 確定變更的影響有多容易? \\
         & 軟體如何處理不同的產品版本? \\
        \hline
        改進現有生產流程有多容易? & 更新元數據有多容易? \\
         & 確定變更的影響有多容易?\\
         & 軟體如何因應不同的生產流程修訂?\? \\
        \hline
        尋找與產品或流程相關的數據有多容易? & 要找出生產編號有多容易? \\
         & Odoo 如何產生效能數據? \\
         & 軟體如何呈現效能變化升級? \\
        \hline
    \end{tabular}
\end{table}




\newpage
\chapter{THE ODOO SOFTWARE} 
\pagenumbering{arabic} %設定頁號阿拉伯數字
\setcounter{page}{32}  %設定頁數
\begin{center}
\fontsize{18}{16}\selectfont \textbf{ODOO 軟體}\\
\end{center}


\section{Introduction to the Odoo software ODOO軟體簡介}
\fontsize{12}{2.5pt}\sectionef  
{Odoo is a commercial business management software with strong ties to the open source community. Initially started as open source ERP software becoming well received as an affordable and intuitive package that thrived on integration and expandability. Since then, as the company experienced accelerated growth, it shifted their business model to include an enterprise paid version as well as an online service.}\\[1pt]

\fontsize{12}{2.5pt}\sectionef 
{本論文的目的是透過分析包含所述整合的不同概念和動態,找出使用現成的 Odoo 軟體可以實現 PLM+MES 系統的程度,並應用一個虛構的場景來確定這些概念是否以及哪些概念包含在此打包解決方案中。}\\[15pt]

\fontsize{12}{2.5pt}\sectionef
{To contextualize, the Odoo software differs from other solutions in the market substantially both in implementation and business model. To summarize, the Odoo software 
was originated as an open-source ERP software as oppose to a PLM or MES software and as such its availability and modularity are reasonably expanded. It goes without saying that the counter point for this that its usability in the field of PLM or MES is uncertain hence the value of this work.}\\[1pt]

\fontsize{12}{2.5pt}\sectionef  
{如同 2.2 節中所提到的,現代 ERP 系統通常是模組化的,並且在以 Odoo 為例,由於數量驚人,這種模組化尤為明顯
由社區開發的模組以及公司開發的模組提供的擴展模組高度整合。 這種可擴展性使得該軟體如此重要回到 PLM+MES 整合的主題,因為現有的 PLM 模組以及其製造模組中具有顯著的 MES 功能。}\\[15pt]

\fontsize{12}{2.5pt}\sectionef  
{Within the scope of this thesis, the objective is to utilize this software on the management of the previously mentioned fictional company and draw conclusions regarding how effective the integration of PLM and MES is already present within this system.  }\\[1pt]

\fontsize{12}{2.5pt}\sectionef 
{在本論文的範圍內,目標是利用該軟體進行管理前面提到的虛構公司,並就其有效性得出結論PLM 和 MES 的整合已經存在於該系統中。}\\[15pt]

\subsection{How it works 如何運作 }
\fontsize{12}{2.5pt}\sectionef 
 {The software can be installed in most x86 computers and it supports several operating
systems including windows and all the main Linux distributions.}\\[1pt]

\fontsize{12}{2.5pt}\sectionef  
{該軟體可以安裝在大多數x86電腦上,並且支援多種操作
系統包括 Windows 和所有主要的 Linux 發行版。}\\[15pt]

\fontsize{12}{2.5pt}\sectionef 
 {Ideally, the Odoo software is installed in a computer connected to a local area networkand starts a SQLdatabase that holds all the necessary information and files produced by thebusiness (Figure 16). Said computer works essentially as a server and accessed via abrowser by the other machines present in the network. This computer can be a dedicatedserver or a working desktop in use, but it is important to remember that it must remain ONand connected throughout the entire time the software is required to function.}\\[1pt]

\fontsize{12}{2.5pt}\sectionef  
{理想情況下,Odoo 軟體安裝在連接到區域網路的電腦上並啟動一個 SQL 資料庫,其中包含由該程式產生的所有必要資訊和文件業務(圖 16)。 所述計算機本質上作為伺服器工作並透過網路中其他機器的瀏覽器。 這台計算機可以是專用的伺服器或正在使用的工作桌面,但重要的是要記住它必須保持開啟狀態並且在軟體需要運作的整個過程中都保持連線。}
\\[15pt]


\begin{figure}[hbt!]
\begin{center}
\includegraphics[width=15cm]{16}
\caption{\Large  Function Diagram of Odoo configuration A Odoo配置A功能圖}\label{fig.16}
\end{center}
\end{figure}


\fontsize{12}{2.5pt}\sectionef 
 {Another option is to use the hosting service provided by Odoo SA (Figure 17). In this case
the system would be hosted by them and data would be stored in their cloud. This is a good
fit for many small businesses specially if they are particularly fond of the website related
modules (used to build and manage web sites and e-stores). It is however network dependent
which may pose a problem in some instances.}\\[1pt]

\fontsize{12}{2.5pt}\sectionef  
{另一個選擇是使用 Odoo SA 提供的託管服務(圖 17)。 在這種情況下
該系統將由他們託管,數據將儲存在他們的雲端中。 這是一個很好的
適合許多小型企業,特別是如果他們特別喜歡相關網站
模組(用於建立和管理網站和電子商店)。 然而,它依賴於網絡
這在某些情況下可能會造成問題。}
\\[15pt]


\begin{figure}[hbt!]
\begin{center}
\includegraphics[width=15cm]{17}
\caption{\Large  Function Diagram of Odoo configuration B Odoo配置B功能圖}\label{fig.17}
\end{center}
\end{figure}

\fontsize{12}{2.5pt}\sectionef 
 {Users essentially interact with the system through the graphical user interface (GUI) and
use it to access the different modules available as need by a per user basis. This means that
restrictions can be applied to different users in order to maintain control over the different
aspects of the business activity, e.g., accountants would get access to accounting module,
sales module and inventory module but they would be restricted from the manufacturing
module. This sort of restriction guarantees control over the processes only to the proper
employees.}\\[1pt]

\fontsize{12}{2.5pt}\sectionef  
{使用者本質上是透過圖形使用者介面(GUI)與系統互動的使用它可以根據每個用戶的需要存取可用的不同模組。 這意味著可以對不同的使用者套用限制,以維持對不同使用者的控制業務活動的各個方面,例如會計師可以存取會計模組,銷售模組和庫存模組,但它們將被限制在製造中模組。 這種限制保證了只有適當的人才能控制流程僱員。}
\\[15pt]


\fontsize{12}{2.5pt}\sectionef 
 {Within said GUI the different modules appear as app icons (Figure 18) and, from the getgo, the company has available a reasonable selection of well-integrated applications not to mention a vast app store filled with community made modules.}\\[1pt]

\fontsize{12}{2.5pt}\sectionef  
{在所述GUI 中,不同的模組顯示為應用程式圖標(圖18),並且從一開始,該公司就提供了合理的整合良好的應用程式選擇,更不用說充滿社區製作模組的龐大應用程式商店。}
\\[15pt]


\begin{figure}[hbt!]
\begin{center}
\includegraphics[width=15cm]{18}
\caption{\Large Screenshot of GUI from Odoo in configuration B 配置 B 中 Odoo 的 GUI 螢幕截圖}\label{fig.18}
\end{center}
\end{figure}

\subsection{Odoo’s view on manufacturing: Odoo 對製造業的看法:}


\fontsize{12}{2.5pt}\sectionef 
 {Odoo considers that the responsibilities regarding manufacturing of anything is distributed throughout different company departments, each of which is responsible for
specific file types and dealt with using specific apps (Table 2). From the perspective of PLM this is very positive because as mentioned by (Saaksvuori and Immonen, 2008) about User privilege management – the PLM system is used to define information access and maintenance rights. The PLM system defines the people who can create new information or make, check and accept changes, and those who are allowed only to view the information or documents in the system. user privilege management is usually a challenge when regardingintegration of PLM with other systems.}\\[1pt]

\fontsize{12}{2.5pt}\sectionef  
{Odoo 認為任何產品製造的責任都是分佈在公司不同部門,各部門負責特定的文件類型並使用特定的應用程式進行處理(表 2)。 從PLM的角度來看這是非常正面的,因為正如(Saaksvuori 和 Immonen,2008)關於使用者所提到的權限管理-PLM系統用於定義資訊存取和維護權利。 PLM 系統定義了可以建立新資訊或進行、檢查和接受更改,以及那些僅被允許查看資訊或系統中的文件。 使用者權限管理通常是一個挑戰PLM 與其他系統的整合。}
\\[15pt]


\begin{table}[htbp]
    \centering
    \begin{tabular}{|c|c|}
        \hline
         Department 部門& Documents/Apps 文件/應用程式\\
        \hline
        Engineering  工程& CAD BOM CAD 和物料清單 \\
        Manufacturing Engineering  製造工程& Routings, Worksheets, Workcenters 工藝路線、工作表、工作中心\\
       Purchase/Procurement 採購/採購& Procurement order, Request for quotation 採購訂單、詢價單\\
       Inventory Operators 庫存操作員&Receipt, Barcode 收據、條碼 \\
       Manufacturing Foreman 製造工長& Manufacturing order, Planning 製造訂單、計劃 \\
       Manufacturing Operators 製造經營者& Work order工作指示 \\
       Inventory Operators 庫存操作員& Delivery送貨 \\
       Quality 品質& Alert, Analysis, Control points 警報、分析、控制點\\、
      Engineering  工程& Engineering change order工程變更單 \\
      Maintenance 維護& Preventive/Corrective 預防/糾正\\
        \hline
    \end{tabular}
\end{table}


\fontsize{12}{2.5pt}\sectionef 
 {From Odoo’s perspective in the beginning of any usual manufacturing process, the first
step will be the engineers designing the product usually using a CAD software. Once that is
done, they will create a Bill of materials (BOM) this is a list of components or materials
necessary to produce the product. At this point the focus goes to the manufacturing process
itself.}\\[1pt]

\fontsize{12}{2.5pt}\sectionef  
{從 Odoo 的角度來看,在任何通常的製造過程開始時,第一個
步驟是工程師通常使用 CAD 軟體設計產品。 一旦那是
完成後,他們將創建物料清單 (BOM),這是組件或材料的列表
生產產品所必需的。 此時重點轉向製造過程
本身。}
\\[15pt]

\fontsize{12}{2.5pt}\sectionef 
 {The software view of process is focused on routings, worksheets and work centers this is
done by the manufacturing engineering team. A routing is a set of steps a product goes
through for production. Worksheets are the instructions for the manufacturing operator, and
work centers are the places where the production is being conducted. Odoo considers that
these are the requirements for putting engineers plans in motion}\\[1pt]

\fontsize{12}{2.5pt}\sectionef  
{流程的軟體視圖側重於工藝路線、工作表和工作中心,這是
由製造工程團隊完成。 製程路線是產品運作的一組步驟
透過進行生產。 工作表是給製造操作員的說明,並且
工作中心是進行生產的地方。 奧杜認為
這些是實施工程師計劃的要求}
\\[15pt]

\fontsize{12}{2.5pt}\sectionef 
 {A procurement department will be responsible for requesting for quotations (RFQ) or
purchase orders (PO). Inventory operators take care of receipts based on those POs, which is
usually done using a barcode application within Odoo. As explained in the first section of
this chapter Odoo is primarily an ERP system and it is at this point that it is possible to notice
some ERP centric characteristics like the focus on inventory and management of resources.
This will be further analyzed in the following sections, but it is fair to point out that those
RFQ and PO are considered items within the data base.}\\[1pt]

\fontsize{12}{2.5pt}\sectionef  
{採購部門將負責索取報價(RFQ)或
採購訂單 (PO)。 庫存操作員根據這些 PO 處理收貨,即
通常使用 Odoo 中的條碼應用程式完成。 正如第一節所解釋的
本章 Odoo 主要是一個 ERP 系統,此時可以注意到
一些以 ERP 為中心的特徵,例如專注於庫存和資源管理。
這將在以下各節中進一步分析,但公平地指出,這些
RFQ 和 PO 被視為資料庫中的項目。}
\\[15pt]

\fontsize{12}{2.5pt}\sectionef 
 {Only when you have the design the process and the materials required Odoo considers
manufacturing possible. Then the manufacturing foreman will create a manufacturing order
(MO) and manage the planning of the manufacturing operators through work orders (WO)
and work centers. Then the manufacturing operators can start production following a work
order. After the products are produced, they automatically appear in the inventory database
which alongside packaging and delivery is managed by the Inventory department.}\\[1pt]

\fontsize{12}{2.5pt}\sectionef  
{只有當您設計了 Odoo 考慮的流程和所需材料時
製造成為可能。 然後製造工長將創建製造訂單
(MO) 並透過工單管理製造操作員的計畫 (WO)
和工作中心。 然後製造操作員可以在工作結束後開始生產
命令。 產品生產出來後,自動出現在庫存資料庫中
與包裝和交付一起由庫存部門管理。}
\\[15pt]


\fontsize{12}{2.5pt}\sectionef 
 {Odoo considers that quality team is responsible for assign control/check points as well as
identify possible issues within the product or production. These quality control check points
are very interesting from the MES perspective because it represents valuable production data
that is collected in real time as production occurs, i.e., it is possible to assign a dimension
check after the production of every piece where the machinist will fill in the dimensions to
track quality over time.}\\[1pt]

\fontsize{12}{2.5pt}\sectionef  
{Odoo 認為品質團隊負責分配控制/檢查點以及
識別產品或生產中可能存在的問題。 這些品質控制檢查點
從 MES 的角度來看非常有趣,因為它代表了有價值的生產數據
在生產發生時即時收集,即可以分配維度
每件產品生產完成後進行檢查,機械師將填寫尺寸
隨著時間的推移跟踪品質。}
\\[15pt]


\fontsize{12}{2.5pt}\sectionef 
 {If it's a problem of design or if there is possibility for improvement an engineering change
order (ECO) can be issued. This falls back to the hands of the manufacturing engineering
team and will focus on updating documents and the BOM. The ECO is the heart of how Odoo
deals with tracking change within the system. That is key when regarding PLM and in fact is
the focus of the Odoo application called PLM. To which lengths said application is capable
to perform is the subject of the next section.}\\[1pt]

\fontsize{12}{2.5pt}\sectionef  
{如果是設計問題或是否有改進工程變更的可能性
可以發出訂單(ECO)。 這又回到了製造工程的手中
團隊將專注於更新文件和 BOM。 ECO 是 Odoo 的核心
處理追蹤系統內的變化。 對於 PLM 而言,這是關鍵,事實上
Odoo 應用程式的焦點稱為 PLM。 所述應用程式能夠達到什麼長度
執行是下一節的主題。}
\\[15pt]

\subsection{The information structure of Odoo Odoo的資訊結構 }

\fontsize{12}{2.5pt}\sectionef 
 {Each module focuses in the manipulation of specific object-oriented classes that hold
metadata within the database. These are the virtual Items that are responsible for virtualizing
the aspects of the product lifecycle as referred by in (Section 3.1). Different types of items
have different types of accounts and hold different sorts of data, i.e., a product item is
representative of a certain product and holds metadata that is relevant to its interactions and
use as well as links to other possible items that are closely relevant like their responsible user
or the bill of materials necessary to its manufacturing. Odoo them makes all that information
accessible and interactable through its browser interface (Figure 19 and Figure 20). For the
sake of consistency this document will refer to specific item representations (E.g. Bolt) as
‘item’ and refer to a type of item (Product) as ‘item class’.}\\[1pt]

\fontsize{12}{2.5pt}\sectionef  
{每個模組都專注於特定物件導向類別的操作,這些類別包含
資料庫內的元資料。 這些都是負責虛擬化的虛擬項
(第 3.1 節)中提到的產品生命週期的各個面向。 不同類型的物品
擁有不同類型的帳戶並保存不同類型的數據,即產品項目是
代表某個產品並保存與其互動相關的元數據
使用以及指向其他可能的項目的鏈接,這些項目與其負責的用戶密切相關
或其製造所需的物料清單。 Odoo 他們製作了所有這些資訊
可透過其瀏覽器介面進行存取和互動(圖 19 和圖 20)。 為了
為了保持一致性,本文檔將特定的項目表示(例如螺栓)稱為
「item」並將某種類型的項目(產品)稱為「項目類別」。}\\[15pt]

\begin{figure}[hbt!]
\begin{center}
\includegraphics[width=15cm]{19}
\caption{\Large Example of Odoo’s interface regarding items Odoo 有關專案的介面範例}\label{fig.19}
\end{center}
\end{figure}

\begin{figure}[hbt!]
\begin{center}
\includegraphics[width=15cm]{20}
\caption{\Large Example of specific item and its metadata as displayed by GUI GUI 顯示的特定項目及其元資料的範例}\label{fig.20}
\end{center}
\end{figure}


\fontsize{12}{2.5pt}\sectionef 
 {Within Odoo, there are several types of those item classes (some holding a lot of metadata
and some holding very little) all with a varying degree of relationships and integration. Since
the scope of this work is limited to the PLM and MES capabilities, the focus is on the items
that are related to it. The following sections will provide short explanations for the main 7
item classes of Odoo’s manufacturing process since its basic understanding is helpful for the
reader to follow the simulation. These are represented in the following diagram (Figure 21).
Other items that are external to the manufacturing procedure will be presented throughout
the simulation.}\\[1pt]

\fontsize{12}{2.5pt}\sectionef  
{在 Odoo 中,這些項目類有多種類型(有些包含大量元資料)
有些持有很少),所有這些都具有不同程度的關係和整合。 自從
這項工作的範圍僅限於 PLM 和 MES 功能,重點是專案
與它相關的。 以下部分將對主要 7 個部分進行簡短說明
Odoo 製造流程的專案類別,因為它的基本了解有助於
讀者跟隨模擬。 這些如下圖所示(圖 21)。
製造過程之外的其他項目將在整個過程中呈現
模擬。}\\[15pt]


\begin{figure}[hbt!]
\begin{center}
\includegraphics[width=15cm]{21}
\caption{\Large Simplified Item relation diagram to the manufacturing of a product X  與產品 X 的製造相關的簡化專案關係圖}\label{fig.21}
\end{center}
\end{figure}



\subsubsection{Product Item 產品項目 }

\fontsize{12}{2.5pt}\sectionef 
 {Every material, component or product is characterized by a PRODUCT type class that is
held and mainly managed within the Inventory application of Odoo. That means that within
the system product production is dependent on the availability of other products that are
either bought as they are or manufactured from another products (Figure 22), i.e., raw
materials are considered products as well, more specifically products that are purchased and
then included in the BOM’s to manufacture other products. This is considered the main item
class since it is both the source and the goal of manufacturing.
}\\[1pt]

\fontsize{12}{2.5pt}\sectionef  
{每種材料、組件或產品都具有一個產品類型類別,該類別是
主要在 Odoo 的庫存應用程式中進行持有和管理。 這意味著在
系統產品的生產取決於其他產品的可用性
要麼按原樣購買,要麼用其他產品製造(圖 22),即原料
材料也被視為產品,更具體地說,是購買和使用的產品
然後包含在 BOM 中以製造其他產品。 這被認為是主要項目
階級,因為它既是製造的源泉,也是製造的目標。}\\[15pt]


\begin{figure}[hbt!]
\begin{center}
\includegraphics[width=15cm]{22}
\caption{\Large  simplified Product relation diagram  簡化的產品關係圖}\label{fig.22}
\end{center}
\end{figure}


\subsubsection{Operation item class and workcenter item class 操作項目類和工作中心項目類}

\fontsize{12}{2.5pt}\sectionef 
 {The operation item is representative of a manufacturing operation that is required to
transform components or raw materials into a product or new component while the
workcenter item represents the place at which the operation takes place, e.g., a sanding wood
will be carried out in a sanding station (Figure 23) that has the proper equipment. The
workcenter is eventually used in Odoo as a time/equipment management tool in its
production planning. Basically, when the production center is at full capacity it puts
following processes on hold or redirects the processes to an alternative workcenter. The
operation item is also responsible for holding the instruction files that are consulted during
production. }\\[1pt]

\fontsize{12}{2.5pt}\sectionef  
{操作項目代表製造操作,需要
將組件或原料轉化為產品或新組件,同時
工作中心項代表進行操作的地點,例如打磨木材
將在具有適當設備的打磨站(圖 23)中進行。 這
workcenter 最終在 Odoo 中用作時間/設備管理工具
計劃生產。 基本上,當生產中心滿載運轉時,
追蹤暫停的流程或將流程重定向到備用工作中心。 這
操作項也負責保存操作過程中查閱的指令文件
生產。}\\[15pt]

\begin{figure}[hbt!]
\begin{center}
\includegraphics[width=15cm]{23}
\caption{\Large Simplified Operation diagram  簡化操作圖}\label{fig.23}
\end{center}
\end{figure}


\subsubsection{The Bill of Materials item class 物料清單項目類別 }

\fontsize{12}{2.5pt}\sectionef 
 {The Bill of Materials is a list of components necessary to build a product. In Odoo,
however, the BOM is best described by what PLM would consider the virtual representation
of the production process. That might seem counter intuitive at first considering the
previously mentioned operation item class, but in fact since the BOM is a compound item it
points directly to all item types necessary to produce the end product (Figure 24). For
example, let’s say that to build a product it is required 3 different parts and 4 different
operations; the BOM of said product would list all of them as well as specify the order in
which these are utilized.}\\[1pt]

\fontsize{12}{2.5pt}\sectionef  
{物料清單是建立產品所需組件的清單。 在奧杜,
然而,BOM 最好透過 PLM 認為的虛擬表示來描述
的生產過程。 乍一看,這似乎違反直覺
前面提到了操作項類,但實際上由於BOM是複合項它
直接指向生產最終產品所需的所有項目類型(圖 24)。 為了
例如,假設要建造一個產品,需要 3 個不同的部件和 4 個不同的部件
營運; 所述產品的 BOM 將列出所有這些產品並指定順序
這些都被利用了。}\\[15pt]

\begin{figure}[hbt!]
\begin{center}
\includegraphics[width=15cm]{24}
\caption{\Large Simplified BOM diagram  簡化的 BOM 圖}\label{fig.24}
\end{center}
\end{figure}

\subsubsection{Manufacturing order item class and work order item class  製造訂單項目類和工作訂單項目類 }

\fontsize{12}{2.5pt}\sectionef 
 {Along the standard items that are considered within Odoo, orders are the ones that
represent commencement within the system. They are signaling that a change is taking place
somehow and somewhere. In the case of a manufacturing order it represents the order to
manufacture N number of specific products using it’s BOM as a base. It is as consequence
of that MO that work orders are automatically generated by Odoo (one for each necessary
operation listed in the BOM) and allocated throughout available necessary workcenters
(Figure 25). }\\[1pt]

\fontsize{12}{2.5pt}\sectionef  
{沿著 Odoo 中考慮的標準項目,訂單是那些
代表系統內的開始。 他們發出信號表明變化正在發生
以某種方式和某處。 在製造訂單的情況下,它代表訂單
使用 BOM 作為基礎製造 N 種特定產品。 其結果就是
該 MO 的工作訂單由 Odoo 自動產生(每個必要的工作訂單一個)
BOM 中列出的操作)並指派到可用的必要工作中心
(圖 25)。}\\[15pt]

\fontsize{12}{2.5pt}\sectionef 
 {The work order is the main form in which the manufacturing operators interact with Odoo,
it presents all the instructions specified by the operation item, as well as control towards its
completion. When a WO takes place the operator signals through the interface its beginning,
its completion and even any quality control check points required while the system keeps
track of timing and performance (Figure 26). Once all WO are done the MO can be declared
done and the materials and components specified in the BOM are consumed and the N copies
of the product is added to inventory. All that makes the work order a central piece as far as
MES is concerned.  }\\[1pt]

\fontsize{12}{2.5pt}\sectionef  
{工單是製造操作員與 Odoo 互動的主要形式,
它呈現了操作項指定的所有指令,以及對其的控制
完成。 當 WO 發生時,操作員透過介面發出開始訊號,
它的完成,甚至是系統保持時所需的任何品質控制檢查點
追蹤時間和性能(圖 26)。 一旦所有 WO 完成,即可宣布 MO
完成並且BOM中指定的材料和組件被消耗並且N份
產品的數量被加到庫存中。 所有這些使得工單成為核心部分
MES很在意。}\\[15pt]

\begin{figure}[hbt!]
\begin{center}
\includegraphics[width=15cm]{25}
\caption{\Large Simplified orders diagram  簡化訂單圖}\label{fig.25}
\end{center}
\end{figure}

\begin{figure}[hbt!]
\begin{center}
\includegraphics[width=15cm]{26}
\caption{\Large Operator interface during the WO  WO 期間的操作員介面}\label{fig.26}
\end{center}
\end{figure}


\subsubsection{ The engineering change order 工程變更單 }

\fontsize{12}{2.5pt}\sectionef 
 {As explained in the beginning of chapter 2 the Odoo management software considers
PLM mainly as a tool for tracking change and improvements. Its application module is
external to the normal flow of manufacturing but acts as an expansion to it. Its focal item
class is the Engineering Change Order (ECO). }\\[1pt]

\fontsize{12}{2.5pt}\sectionef  
{如第 2 章開頭所解釋的,Odoo 管理軟體考慮
PLM 主要作為追蹤變更和改進的工具。 其應用模組為
位於正常製造流程之外,但充當其擴展。 其重點項目
類別是工程變更單(ECO)。}\\[15pt]



\fontsize{12}{2.5pt}\sectionef 
 {An ECO is an item class that outlines the proposed changes to the product or the parts that
would be affected by the change. In other words, is a central information hub for everyone
associated with a given product. }\\[1pt]

\fontsize{12}{2.5pt}\sectionef  
{ECO 是一個項目類別,概述了產品或零件的建議變更
將受到該變化的影響。 換句話說,就是每個人的中央資訊中心
與給定產品相關聯。}\\[15pt]


\fontsize{12}{2.5pt}\sectionef 
 {The idea is to signal the need for change to a product item or a BOM item, hold the files
that are relevant to the change and apply the change or at least signal that the change has been
implemented, all while keeping the history of al the previous changes. All very useful in the
future and serve as a process to streamline product development and help improve
products/production.}\\[1pt]

\fontsize{12}{2.5pt}\sectionef  
{這個想法是表明需要更改產品項目或 BOM 項目,保存文件
與變更相關並應用變更或至少表示變更已經
實施,同時保留所有先前更改的歷史記錄。 一切都非常有用
未來並作為簡化產品開發並幫助改進的流程
產品/生產。}\\[15pt]



\begin{figure}[hbt!]
\begin{center}
\includegraphics[width=15cm]{27}
\caption{\Large Simplified ECO function diagram 簡化的ECO功能圖}\label{fig.27}
\end{center}
\end{figure}
\newpage
\end{document}