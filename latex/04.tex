\chapter{INTRODUCTION TO THE COMPANY AND PRODUCT} 
\pagenumbering{arabic} %設定頁號阿拉伯數字
\setcounter{page}{22}  %設定頁數
\begin{center}
\fontsize{18}{16}\selectfont \textbf{公司及產品介紹}\\
\end{center}
\vspace{1em}

\fontsize{14pt}{2.5pt}\sectionef 
{As one can imagine, one of the unique aspects of this work is its focus in one specific software solution that tend to be quite flexible in terms of ease of implementation to different sorts of business. This is contrary to most use cases regarding PLM implementation where the business case is the constant and the system is built around it. Nonetheless, in order to evaluate Odoo as a PLM+MES tool, it is important to consider an example. The advantage here is that a fictional company can be picked for this end maximizing the perceived effect of the software during a simulation.}\\[5pt]

\fontsize{14pt}{5pt}\sectionef
 {可以想像,這項工作的獨特之處之一是它專注於一個特定的軟體解決方案,該解決方案在易於實施不同類型的業務方面往往非常靈活。這與大多數有關 PLM 實施的用例相反,在這些用例中,業務案例是不變的,系統是圍繞它構建的。儘管如此,為了評估 Odoo 作為 PLM+MES 工具的能力,考慮一個範例很重要。這裡的優點是可以選擇一家虛構的公司來實現這一目的,從而在模擬過程中最大化軟體的感知效果。}\\[15pt]

\fontsize{14pt}{2.5pt}\sectionef 
{It is considering all those previously mentioned systems that, for the sake of exemplification, the theoretical company was organized in the molds of Industry 4.0. This company is a recently founded small case manufacturing company that uses plastic injection molding as their primary mean of production and uses additive manufacturing and fast prototyping as part of their business strategy. As explained in chapter 2 those are great examples of the path that industry is taking regarding innovation where mass production is becoming slowly less important than product variety and time to market.}\\[5pt]

\fontsize{14pt}{5pt}\sectionef
 {考慮到前面提到的所有系統,為了舉例說明,理論公司是按照工業 4.0 的模式組織的。該公司是一家最近成立的小型箱體製造公司,使用塑膠射出成型作為主要生產方式,並使用積層製造和快速原型製作作為其業務策略的一部分。如第 2 章所解釋的,這些都是產業在創新方面所採取的路徑的很好的例子,在這種創新中,大規模生產逐漸變得不如產品品種和上市時間重要。}\\[15pt]

\fontsize{14pt}{2.5pt}\sectionef 
{In order to maximize the tracking of change, most of its business are based on lower production batches on mainly automated machinery. This company focus in the production of injected plastic products and rely heavily in flexible machinery for setting production and prototyping. Having that in mind, it should be simple enough to simulate continuous improvement of both product and process to the extent of the evaluated software. Since this sort of everchanging production is extremely dependent on information management of all kinds, it must prove to be a perfect base for applied PLM+MES.}\\[5pt]

\fontsize{14pt}{5pt}\sectionef
 {為了最大限度地追蹤變化,其大部分業務都基於主要自動化機械上的較低生產批次。該公司專注於注射塑膠產品的生產,並嚴重依賴靈活的機械來進行生產和原型製作。考慮到這一點,它應該足夠簡單,可以在評估的軟體範圍內模擬產品和流程的持續改進。由於這種不斷變化的生產極其依賴各種資訊管理,因此它必須成為應用 PLM+MES 的完美基礎。}\\[15pt]

\fontsize{14pt}{2.5pt}\sectionef 
{In this example the company has already implemented, since its recent foundation, the Odoo software and has taken all the necessary training and steps to its proper use. This allow the removal of the boundaries and limitations that are so common regarding implementation of the PLM+MES system to an already existing business, i.e., dependences on legacy systems administrative resistance to change or integration to old procedures. These are obviously important, but it is not within the scope of this work.}\\[10pt]

\fontsize{14pt}{5pt}\sectionef
 {在這個例子中,該公司自最近成立以來已經實施了 Odoo 軟體,並採取了所有必要的培訓和步驟來正確使用該軟體。這樣可以消除在現有業務中實施 PLM+MES 系統時常見的邊界和限制,即依賴遺留系統管理對更改或整合舊程式的抵制。這些顯然很重要,但不屬於本工作的範圍。}\\[15pt]

\fontsize{14pt}{2.5pt}\sectionef 
{The company aims to produce a completely new product by the end of the year. After doing so, the company improved the process of production for said product. Once there is the need for product improvement, said improvement was performed as well.}\\[10pt]

\fontsize{14pt}{5pt}\sectionef
 {該公司的目標是在今年底前生產出全新的產品。在此之後,該公司改進了該產品的生產流程。一旦產品需要改進,也進行改進。}\\[15pt]

\fontsize{14pt}{2.5pt}\sectionef 
{The following diagram (Figure 9) will be taken into consideration as the path of product development and improvement:}\\[10pt]

\fontsize{14pt}{5pt}\sectionef
 {將考慮下圖(圖9)作為產品開發和改進的路徑:}\\[15pt]

\begin{figure}[hbt!]
\begin{center}
\includegraphics[width=15cm]{9}
\caption{\Large Development diagram 開發圖}\label{fig.9}
\end{center}
\end{figure}

\fontsize{14pt}{2.5pt}\sectionef 
{This path aims to transmit to the reader an iterative approach towards development and improvement. The idea is followed by a product design for which a cycle of prototyping and redesign takes effect until satisfactory result is achieved. Then a similar cycle takes place regarding the production process. At the end of this stage initial development is done and the actual production can begin.}\\[10pt]

\fontsize{14pt}{5pt}\sectionef
 {這條路徑旨在向讀者傳達一種開發和改進的迭代方法。這個想法之後是產品設計,原型製作和重新設計的循環生效,直到獲得滿意的結果。然後在生產過程中會發生類似的循環。在此階段結束時,初步開發完成,實際生產即可開始。}\\[15pt]

\fontsize{14pt}{2.5pt}\sectionef 
{It is at this point that ways of stablishing the continuous improvement is important. In the case of this company, we are only considering two main types of upgrade paths, those being, product upgrade and process upgrade respectively.}\\[10pt]

\fontsize{14pt}{5pt}\sectionef
 {正是在這一點上,建立持續改進的方法很重要。就這家公司而言,我們只考慮兩個主要的升級路徑,分別是產品升級和流程升級。}\\[15pt]

\section{The products and processes 產品和工藝}

\fontsize{14pt}{2.5pt}\sectionef 
{Change and effect are the focus of the PLM+MES implementation as such the subject of said change would ideally be something that could afford a reasonable amount of freedom of design. Although the effects of a well implemented PLM+MES should be substantial even in rigid manufacturing environments, where the change is extremely limited, the system will produce much more perceivable change in an enterprise that thrives in innovation because there will be more opportunities to improve the system and gain feedback.}\\[10pt]

\fontsize{14pt}{5pt}\sectionef
 {變更和效果是 PLM+MES 實施的重點,因此理想情況下,上述變更的主題應該能夠提供合理的設計自由。雖然實施良好的PLM+MES 的效果即使在變化極其有限的嚴格製造環境中也應該是顯著的,但該系統將在創新蓬勃發展的企業中產生更明顯的變化,因為將有更多的機會來改進系統並獲得回饋。}\\[15pt]

\fontsize{14pt}{2.5pt}\sectionef 
{From the perspective of improvement, if you compare a product that is a result from sheet metal stamping (Figure 10) to an equivalent product that is the result of a CNC milling procedure (Figure 11) it is easy to perceive that the CNC milled product is more welcoming to upgrades. While the stamping is low cost (by comparison) it depends on heavy high precision metal dyes that are extremely expensive to produce. This means that the cost of enacting change to it is much higher and thus the effect of a system that thrives on tracking change becomes limited.}\\[10pt]

\fontsize{14pt}{5pt}\sectionef
 {從改進的角度來看,如果將鈑金沖壓的產品(圖 10)與 CNC 銑削加工的同等產品(圖 11)進行比較,很容易看出 CNC 銑削的產品更歡迎升級。雖然沖壓成本較低(相比之下),但它依賴於生產成本極其昂貴的重型高精度金屬染料。這意味著對其進行更改的成本要高得多,因此依靠追蹤更改而蓬勃發展的系統的效果變得有限。}\\[15pt]
\newpage

\begin{figure}[hbt!]
\begin{center}
\includegraphics[width=15cm]{10}
\caption{\Large Example of stamped AK74 pattern rifle receiver (Brownnells.com) 印有AK74圖案的步槍機匣範例 (Brownnells.com)}\label{fig.10}
\end{center}
\end{figure}

\begin{figure}[hbt!]
\begin{center}
\includegraphics[width=15cm]{11}
\caption{\Large Example of milled AK74 pattern rifle receiver (sharpsbros.com) 銑削 AK74 型步槍機匣範例 (sharpsbros.com)}\label{fig.11}
\end{center}
\end{figure}

\fontsize{14pt}{2.5pt}\sectionef 
{In the case of this fictional company, it has been determined that the best way to exemplify the PLM+MES effects would be to have products designed around plastic injection molding. It might seem unintuitive at first to consider this manufacturing procedure, like the stamping procedure previously described, since it too depends on high precision molds during production. However, the main differences between the two is regarding ease of prototyping and the cost of upgrading.}\\[10pt]

\fontsize{14pt}{5pt}\sectionef
 {就這家虛構的公司而言,我們確定體現 PLM+MES 效果的最佳方式是圍繞塑膠射出成型設計產品。乍一看,考慮這種製造過程(如前面描述的沖壓過程)似乎不直觀,因為它在生產過程中也依賴高精度模具。然而,兩者之間的主要區別在於原型設計的難易度和升級成本。}\\[15pt]

\fontsize{14pt}{2.5pt}\sectionef 
{Injection molding is a broad and complex field of engineering that involves a huge variety of materials and methods, little of which is of the concern of this work. It is however relevant to point out that for the most part, the pressures involved in the injection molding are one order of magnitude lower than the when we are dealing with steel; softer materials can be used on their molds like CNC milled aluminum. At the same time, new advancements in the field of additive manufacturing have made possible to prototype plastic parts with much closer physical characteristics to the end result of a injected piece. Sometimes even prototype molds (Figure 12) can be used for a lower volume test runs during process upgrades.}\\[10pt]

\fontsize{14pt}{5pt}\sectionef
 {射出成型是一個廣泛而複雜的工程領域,涉及各種各樣的材料和方法,但本工作只涉及其中很少一部分。然而,需要指出的是,在大多數情況下,射出成型中涉及的壓力比我們處理鋼時的壓力低一個數量級;模具上可以使用較軟的材料,例如 CNC 銑削鋁。同時,積層製造領域的新進步使得塑膠零件原型成為可能,其物理特性與注射件的最終結果更加接近。有時,甚至原型模具(圖 12)也可用於製程升級期間的小批量試運行。}\\[15pt]

\begin{figure}[hbt!]
\begin{center}
\includegraphics[width=15cm]{12}
\caption{\Large Example of injection mold made using a 3D printer (thefabricator.com) 使用 3D 列印機製作射出成型模具的範例 (thefabricator.com)}\label{fig.12}
\end{center}
\end{figure}
\newpage

\fontsize{14pt}{2.5pt}\sectionef 
{Additive manufacturing has become an incredible tool for ultra-flexible production. This mindset of continuous improvement, especially when regarding prototyping and iterative design, is a hallmark of the lean mentality that is so relevant in the modern industry.}\\[10pt]

\fontsize{14pt}{5pt}\sectionef
 {增材製造已成為超靈活生產的不可思議的工具。這種持續改進的心態,特別是在原型設計和迭代設計方面,是與現代工業密切相關的精益心態的標誌。
}\\[15pt]

\fontsize{14pt}{2.5pt}\sectionef 
{As mentioned in the previous section, in this case study it is considered the creation of a new product and its production process by the fictional company. This product consists in a plastic small form factor computer case, composed of 3 different parts (Figure 13) that are expected to be designed and prototyped considering combination of additive manufacturing and CNC milling towards a plastic injection molding production.}\\[10pt]

\fontsize{14pt}{5pt}\sectionef
 {如上一節所提到的,在本案例研究中,它被視為虛構公司的新產品的創造及其生產過程。該產品由一個塑膠小型電腦機殼組成,由 3 個不同的部件組成(圖 13),預計在設計和原型製作時考慮將增材製造和 CNC 銑削相結合進行塑膠射出成型生產。
}\\[25pt]

\begin{figure}[hbt!]
\begin{center}
\includegraphics[width=15cm]{13}
\caption{\Large 3D exploded view of the theoretical product 理論產品的 3D 分解圖}\label{fig.13}
\end{center}
\end{figure}
\newpage

\subsection{Part A A部分}

\fontsize{14pt}{2.5pt}\sectionef 
{PART-A (Figure 14) is the core structure of the computer case. It is expected to comport all the pieces necessary for the proper function of the small form factor computer in question. To this end a raw material A was selected to be Acrylonitrile Butadiene Styrene (ABS) this is an opaque thermoplastic polymer and an engineering grade plastic. It is commonly used to produce electronic parts such as phone adaptors, keyboard keys and wall socket plastic guards.}\\[15pt]

\fontsize{14pt}{5pt}\sectionef
 {PART-A(圖14)是電腦機殼的核心結構。它預計將配備所討論的小型計算機正常運行所需的所有部件。為此,原料 A 被選為丙烯腈丁二烯苯乙烯 (ABS),這是一種不透明的熱塑性聚合物和工程級塑膠。它通常用於生產電子零件,例如電話適配器、鍵盤按鍵和牆壁插座塑膠防護罩。}\\[15pt]

\begin{figure}[hbt!]
\begin{center}
\includegraphics[width=15cm]{14}
\caption{\Large Isometric view of Part A A 部分等距視圖}\label{fig.14}
\end{center}
\end{figure}
\vspace{1cm}

\fontsize{14pt}{2.5pt}\sectionef 
{The main reasons for choosing this material specifically are its toughness, its good dimensional stability (resistance to change dimensions after cooling), its high impact resistance and surface hardness. Finally, it is also commonly available in the form of 3D printing filament for extrusion 3D printers which should prove to be quite useful during prototyping.}\\[15pt]

\fontsize{14pt}{5pt}\sectionef
 {特別選擇這種材料的主要原因是其韌性、良好的尺寸穩定性(冷卻後不易改變尺寸)、高抗衝擊性和表面硬度。最後,它通常也以用於擠出 3D 列印機的 3D 列印絲的形式提供,這在原型製作過程中應該非常有用。}\\[15pt]
\newpage

\subsection{Part B and C     B 部分和 C 部分}

\fontsize{14pt}{2.5pt}\sectionef 
{Parts B and C are lids that should snap into place, closing the system. These are very simple pieces and require a certain level of elasticity so it can deform to assure a screwless assembly. These two identical parts are going to be made with Thermoplastic Polyurethane (TPU), because of its elastic nature and great tensile and tear strength. This sort of polymer is often used to produce parts that demand a rubber-like elasticity. TPU performs well at high temperatures and is commonly used in power tools, cable insulations and sporting goods. Finally, TPU is also available in the form of filament for 3D printers which, for the simulation, will be used for prototyping.}\\[15pt]

\fontsize{14pt}{5pt}\sectionef
 {B 和 C 部分是蓋子,應卡入到位,關閉系統。這些都是非常簡單的零件,需要一定程度的彈性,以便可以變形以確保無螺絲組裝。這兩個相同的部件將由熱塑性聚氨酯 (TPU) 製成,因為它具有彈性、拉伸強度和撕裂強度。這種聚合物通常用於生產需要類似橡膠彈性的零件。 TPU 在高溫下表現良好,常用於電動工具、電纜絕緣材料和體育用品。最後,TPU 還可以以細絲的形式用於 3D 列印機,用於模擬,用於原型製作。}\\[15pt]

\begin{figure}[hbt!]
\begin{center}
\includegraphics[width=15cm]{15}
\caption{\Large Part B and C     B 部分和 C 部分}\label{fig.15}
\end{center}
\end{figure}

\subsection{Molds 模具}

\fontsize{14pt}{2.5pt}\sectionef 
{Ideally all molds should be made of steel, for longevity of the mold and product quality. That being said, the injected plastics that are being selected for all parts are not so pressure dependent and their forms are not so complex, so it is assumed that aluminum molds made with a precision CNC machining should suffice to produce said parts.}\\[10pt]

\fontsize{14pt}{5pt}\sectionef
 {理想情況下,所有模具應由鋼製成,以確保模具的使用壽命和產品品質。話雖如此,為所有零件選擇的注射塑膠並不那麼依賴壓力,而且它們的形狀也不是那麼複雜,因此假設用精密 CNC 加工製造的鋁模具應該足以生產所述零件。}\\[15pt]

\fontsize{14pt}{2.5pt}\sectionef 
{It is also assumed that all molds are simple enough to be prototyped using 3D printing. Although this is not always true, it was determined representative enough for this simulation. The type of material used in those prototypes is high temperature resign cured using an SLA 3DPrinter. Additionally, the mold will be considered the main physical aspect to be developed when regarding the production process because it something that directly affects the production as well as something that can be produced in house and tracked as a product would.}\\[10pt]

\fontsize{14pt}{5pt}\sectionef
 {也假設所有模具都足夠簡單,可以使用 3D 列印來製作原型。儘管這並不總是正確的,但它對於本次模擬來說具有足夠的代表性。這些原型中使用的材料類型是使用 SLA 3D 列印機進行高溫重新固化的。此外,在生產過程中,模具將被視為要開發的主要物理方面,因為它直接影響生產,並且可以像產品一樣在內部生產和追蹤。}\\[15pt]

\section{What is analized during the simulation 模擬過程中分析了什麼}

\fontsize{14pt}{2.5pt}\sectionef 
{Taking into consideration the diagram, shown in Figure 9, as well as the main aspects of a successful integration of PLM and MES as described in the section 3.1, this experiment aims to produce commentary regarding the following relevant questions in Table 1.}\\[10pt]

\fontsize{14pt}{5pt}\sectionef
 {考慮到圖 9 所示的圖表,以及第 3.1 節中描述的 PLM 和 MES 成功整合的主要方面,本實驗旨在對錶 1 中的以下相關問題進行評論。}\\[15pt]

\begin{center}
 \fontsize{16}{16}\selectfont \textbf{Table 1 Summary of questions to be answered}\\
\fontsize{16}{16}\selectfont \textbf{表1 需回答的問題總表}\\
\end{center}

\begin{figure}[hbt!]
\begin{center}
\includegraphics[width=15cm]{15-2}
\end{center}
\end{figure}

\newpage



\begin{table}[htbp]
    \centering
    \label{tab:example}
    \begin{tabular}{|c|c|}
        \hline
        類別 & 問題 \\
        \hline
        軟體如何處理物品? & 是否代表了產品生命週期的所有面向? \\
         & 這些項目的表現如何?\\
         \hline
        創造一個全新的產品有多容易? & 產品的描述方式? \\
         & 產品如何整合和引用相關文件?\\
         & 改變一個會影響另一個嗎? \\
         \hline
       創造一個是多麼容易全新的生產流程? & 其過程是如何描述的? \\
         & 該流程如何整合和參考其生產的產品?\\
         & 改變一個會影響另一個嗎? \\
         \hline
        改進現有產品有多容易? & 更新元數據有多容易? \\
         & 確定變更的影響有多容易? \\
         & 軟體如何處理不同的產品版本? \\
        \hline
        改進現有生產流程有多容易? & 更新元數據有多容易? \\
         & 確定變更的影響有多容易?\\
         & 軟體如何因應不同的生產流程修訂?\? \\
        \hline
        尋找與產品或流程相關的數據有多容易? & 要找出生產編號有多容易? \\
         & Odoo 如何產生效能數據? \\
         & 軟體如何呈現效能變化升級? \\
        \hline
    \end{tabular}
\end{table}



