\chapter{INTRODUCTION} 
\pagenumbering{arabic} %設定頁號阿拉伯數字
\setcounter{page}{1}  %設定頁數
\begin{center}
\fontsize{18}{16}\selectfont \textbf{介紹}\\
\end{center}
\fontsize{12pt}{2.5pt}\sectionef
\section{Objective目的}
\fontsize{12}{2.5pt}\selectfont {The thesis has the objective of finding out how far PLM+MES system can be implemented 
by using the readily available Odoo software by analyzing the different concepts and 
dynamics that would consist said integration and they apply a fictional scenario to determine 
if and which of those concepts are included within this packaged solution.}\\[1pt]

\fontsize{12}{2.5pt}\selectfont {本論文的目的是透過分析包含所述整合的不同概念和動態,找出使用現成的 Odoo 軟體可以實現 PLM+MES 系統的程度,並應用一個虛構的場景來確定這些概念是否以及哪些概念包含在此打包解決方案中。}\\[15pt]

\fontsize{12}{2.5pt}\selectfont {To contextualize, the Odoo software differs from other solutions in the market 
substantially both in implementation and business model. To summarize, the Odoo software 
was originated as an open-source ERP software as oppose to a PLM or MES software and as 
such its availability and modularity are reasonably expanded. It goes without saying that the 
counter point for this that its usability in the field of PLM or MES is uncertain hence the 
value of this work.}\\[1pt]

\fontsize{12}{2.5pt}\selectfont {從具體情況來看,Odoo 軟體在實施和業務模式方面與市場上的其他解決方案有很大不同。 總而言之,Odoo 軟體最初是一種開源 ERP 軟體,而不是 PLM 或 MES 軟體,因此其可用性和模組化得到了合理擴展。 不言而喻,與之相反的是,它在 PLM 或 MES 領域的可用性是不確定的,因此這項工作的價值。}\\[15pt]

\fontsize{12}{2.5pt}\selectfont {Specifically, from the perspective of small manufacturing business and startups, the idea 
of an all-around ERP that implements a PLM-MES system is extremely valuable. Although
ERP systems are somewhat available, they rarely venture deep enough into manufacturing to 
expand into PLM or MES solutions. In addition, the other direction is also relevant since 
PLM solutions tend to not have the expandability of an ERP which usually means that any 
integration requires specialized ad-hoc work.}\\[1pt]

\fontsize{12}{2.5pt}\selectfont {具體來說,從小型製造企業和新創企業的角度來看,實施PLM-MES系統的全能ERP的想法非常有價值。 儘管 ERP 系統在一定程度上可用,但它們很少深入製造業以擴展到 PLM 或 MES 解決方案。 此外,另一個方向也相關,因為 PLM 解決方案往往不具備 ERP 的可擴展性,這通常意味著任何整合都需要專門的臨時工作。}\\[15pt]

\fontsize{12}{2.5pt}\selectfont {Although modifying the software do not fall within the scope of this work, the fact that the software has an open-source community version means that adapting the software even to the most specific cases may prove to be easier and economical barriers for adopting lower,further emphasizing the possible utility of this software in the context of small business.}\\[1pt]

\fontsize{12}{2.5pt}\selectfont {儘管修改軟體不屬於這項工作的範圍,但該軟體具有開源社群版本這一事實意味著,甚至使該軟體適應最具體的情況也可能被證明是更容易和更經濟的採用更低、更進一步的障礙。強調該軟體在小型企業中的可能實用性。}\\[15pt]

\fontsize{12}{2.5pt}\selectfont {具體來說,從小型製造企業和新創企業的角度來看,實施PLM-MES系統的全能ERP的想法非常有價值。 儘管 ERP 系統在一定程度上可用,但它們很少深入製造業以擴展到 PLM 或 MES 解決方案。 此外,另一個方向也相關,因為 PLM 解決方案往往不具備 ERP 的可擴展性,這通常意味著任何整合都需要專門的臨時工作。}\\[15pt]
\newpage
\fontsize{12}{2.5pt}\selectfont {Ultimately, the thesis will give theoretical and practical advices on how to further exploit this system. It will also lay the ground for future works on the Odoo software and checks on how the solution is performing by identifying specific key aspects of PLM-MES integration and implementation}\\[1pt]

\fontsize{12}{2.5pt}\selectfont {最終,本文將為如何進一步利用該系統提供理論和實務建議。 它還將為 Odoo 軟體的未來工作奠定基礎,並透過確定 PLM-MES 整合和實施的具體關鍵方面來檢查解決方案的性能}\\[15pt]
\section{Structure結構 }
\fontsize{12}{2.5pt}\selectfont {This work could be a reference for an actual implementation of the described solution in 
small manufacturing enterprises and it can be treated as introductory material to PLM-MES 
and their implementation, as well as first principles and review of the current state of the 
Odoo software regarding it. To such end, this thesis presents the following structure:}\\[1pt]

\fontsize{12}{2.5pt}\selectfont {這項工作可以為小型製造企業中所描述的解決方案的實際實施提供參考,並且可以將其視為 PLM-MES 及其實施的介紹材料,以及 Odoo 軟體的首要原則和當前狀態的回顧它。 為此,本文提出以下結構:}\\[15pt]
\begin{enumerate}[{\hspace{0.5em}\textbullet}]
\fontsize{12}{2.5pt}\selectfont
            \item Chapter 1 - Introduction to this work and its objectives. Furthermore, it provide a 
succinct explanation of why this software solution requires this sort of analysis in the 
first place and how it was be structured.\\
第 1 章 - 介紹這項工作及其目標。 此外,它還簡要解釋了為什麼該軟體解決方案首先需要進行此類分析以及它是如何建構的。
\item Chapter 2 – This chapter introduce the basic theoretical background to PLM, MES, 
ERP and Industry 4.0. These are presented in order to create the grounds to a 
meaningful contribution in this kind of analysis as well as providing meaningful 
context for its implementation in case the reader is a small business representative.\\
第 2 章 – 本章介紹 PLM、MES、ERP 和工業 4.0 的基本理論背景。 提出這些內容是為了為此類分析做出有意義的貢獻奠定基礎,並為讀者是小型企業代表的情況下的實施提供有意義的背景。
\item Chapter 3 – This chapter is all about the integration between PLM and MES systems 
as discussed by previous works and as was be analyzed in this work. This is useful to 
stablish the concepts and dynamics that are the subject when analyzing the Odoo
software.\\
第 3 章 – 本章主要介紹 PLM 和 MES 系統之間的集成,如先前的工作所討論的和本工作中所分析的。 這對於在分析 Odoo 軟體時確定主題的概念和動態很有用。
\item Chapter 4 – Introduction to the fictional company and products chosen in the molds 
of Industry 4.0 to be used in the further analysis and evaluation of the Odoo software.\\
第 4 章 – 介紹在工業 4.0 模型中選擇的虛構公司和產品,用於進一步分析和評估 Odoo 軟體。
\item  Chapter 5 – The introduction to the Odoo software as well as a more in-depth
explanation of its use and functionalities. The description of the experimentation of 
the Odoo software taking in consideration all the previous chapters.\\
第 5 章 – Odoo 軟體簡介以及對其使用和功能的更深入解釋。 考慮到前面所有章節的 Odoo 軟體實驗描述
\item Chapter 7 - Conclusions The last chapter describes the takeaways of the work: how a 
medium enterprise can improve its processes through an informed use of a 
PLM+MES system implemented using the Odoo software.\\
第 7 章 - 結論 最後一章介紹了工作的要點:中型企業如何透過明智地使用使用 Odoo 軟體實施的 PLM+MES 系統來改善其流程。

\end{enumerate}
\renewcommand{\baselinestretch}{0.5} %設定行距
